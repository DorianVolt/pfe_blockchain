Les échanges financiers sur Internet reposent presque exclusivement via les institutions financières, qui
agissent comme tiers de confiance pour le traitement des paiements électroniques.
En 2008, le \textit{whitepaper} Bitcoin \cite{Bitcoin} a partagé une solution permettant à deux parties d'échanger de
la monnaie électronique. La particularité de cette solution est la suppression de ce modèle de confiance par
l'ajout d'une preuve cryptographique. Cela permet de s'émanciper de la centralisation exercée par ces 
institutions et de tendre vers la décentralisation.
\newline

La structure de donnée sous-jacente est la \textit{\gls{blockchain}}, une base de données distribuée constituée 
d’une chaîne de blocs liés et sécurisés par des hachés cryptographiques. Une des propriétés 
d'une \gls{fonction de hachage cryptographique} est qu'une modification de l'entrée modifie la sortie, le haché.
D'où le fait qu'une \textit{\gls{blockchain}} est considérée comme immuable , chaque bloc (sauf le premier) 
est lié au bloc précédent car il contient le haché de ce dernier. Toute tentative de modifier un bloc antérieur 
affecterait tous les blocs suivants, créant ainsi une incohérence dans la chaîne.
\newline

Dans un système centralisé, toutes les transactions sont enregistrées dans une base de données unique gérée 
par une entité centrale telle qu'une banque. Cette entité est responsable de l'intégrité des données
et les usagers lui font confiance que toute modification involontaire ou malveillante sera détectée. 
Dans le cas de la \textit{\gls{blockchain}}, elle est distribuée, un réseau de noeuds connectés travaillent ensemble pour valider et enregistrer les blocs. 
Un noeud est un périphérique connecté à un réseau pair à pair qui stocke une copie de la \textit{\gls{blockchain}} et participe à la validation 
et à la propagation des blocs. En raison de sa nature distribuée, les noeuds peuvent valider et ajouter un bloc sans avoir 
recours à un tiers de confiance, ce qui accrût son caractère décentralisé.
\newline

Depuis, le concept de \textit{\gls{blockchain}} a grandi en popularité et il existe aujourd'hui un grand nombre de \textit{\gls{blockchain}s}.
Une problématique est apparue : les utilisateurs ont voulu échanger des jetons provenant de \textit{\gls{blockchain}s} différentes mais ces
mêmes \textit{\gls{blockchain}s} ne supportaient pas les mêmes protocoles.
D'où l'implémentation de protocoles d'échanges de jetons inter-\textit{\gls{blockchain}} qui a rajouté un maillon supplémentaire à sécuriser, 
ce dernier étant un vecteur d'attaque privilégié.
L'objectif de ce rapport est de dresser un état de l'art de ces protocoles en deux temps : les protocoles d'échanges sur les 
plateformes centralisées puis les protocoles d'échanges décentralisés.
