Le swap d’actifs entre différentes blockchains est une propiété fortement rechechée dans le secteur de la
blockchain et des echanges de crypto-actifs. En effet, dans un contexte où l’adoption de la technologie est
grandissante \cite{evolutionCrypto2022} il est largement souhaités et utiles de mettre en place des solutions d’echanges entre chaines,
permettant ainsi aux utilisateurs de transférer des actifs d’une blockchain à une autre sans avoir à passer par
un échange centralisé. Cela peut être très utile pour les utilisateurs qui souhaitent échanger des actifs qui
ne sont pas disponibles sur leur blockchain d’origine ou qui souhaitent simplement utiliser une blockchain
différente pour des raisons de sécurité ou de confidentialité. Cependant, les swaps d’actifs entre différentes
blockchains posent des problèmes de sécurité et de confiance car il est difficile de garantir que les actifs seront
transférés en toute sécurité et que les utilisateurs ne seront pas victimes d’une fraude ou d’une arnaque. Il est
donc important de mettre en place des solutions sécurisées et fiables pour les swaps d’actifs entre différentes
blockchains.


\begin{itemize}
    \item Def blockchain
    \item Def échange inter-blockchains
    \item notre définition de  centralisé/décentralisé
\end{itemize}
