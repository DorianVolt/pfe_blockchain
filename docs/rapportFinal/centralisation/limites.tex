%Auteur : Romain TESTUD
Pour conclure sur les échanges centralisés, nous avons pu voir que le fonctionnement des plate-formes est relativement opaque. 
En effet, les protocoles employés ne sont pas ou peu explicités, nous avons trouvé peu de documentation technique au cours de nos recherches. 
La majeure partie de la documentation que nous avons pu trouver concernant les plate-formes d'échanges centralisé sont à destination des utilisateurs de ces plate-formes, donc à des personnes n'ayant pas forcément un bagage de connaissance dans les \textit{\gls{blockchain}s}. 
C'est pour cela que nombreuses de nos sources concernant les plate-formes centralisés sont issues de réseaux sociaux ou d'articles web à destination de lecteurs spécialisés dans le domaine. 
La solution centralisée au fonctionnement le plus ouvert au public que nous avons trouvé est \textit{\gls{Wormhole}}, il s'avère que ce protocole à aussi été victime d'une attaque de grande envergure. \\
Nous pouvons ainsi nous questionner sur la présence de l'intermédiaire de confiance dans l'échange centralisé. 
En effet, la majeure partie des attaques observés sur des \textit{Bridges} ou plus généralement sur des protocoles d'échanges centralisés, résultent de problèmes d'implémentation dans le code de ces protocoles. 
Le tiers de confiance est donc un point d'entrée pour les pirates vers les \gls{actif}s des utilisateurs. 
Cette menace constante a menée à des recherches visant à soutirer cet intermédiaire des échanges inter-\gls{blockchain}, donc d'employer des moyens décentralisés pour résoudre cette problématique. 