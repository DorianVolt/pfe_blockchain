% Auteur : Romain TESTUD
\begin{frame}{Plate-formes d'échanges Centralisés}
    \begin{block}{Plate-formes privées d'échanges de données sur blockchain}
        \begin{itemize}
            \item Possibilité d'échanges inter-blockchains par l'utilisation de bridges.
            \item Présence d'un intermédiaire entre les utilisateurs.
        \end{itemize}
    \end{block}
\pause
    \begin{block}{Méthode de l'Order Book}
        \begin{itemize}
            \item Dépôt des fonds dans un porte monnaie de la plate-forme. 
            \item Émission d'IOU\footnotemark.
            \item Échange des IOU contre les token demandés.
        \end{itemize}
    \end{block}
    \footnotetext{IOU (I Owe You) : Reconnaissance de dette de la plate-forme envers l'utilisateur}
    %L'utilisateur place un ordre d'achat, la plateforme trouve un vendeur, l'échange s'effectue
\end{frame}

\begin{frame}{Avantages et inconvénients}
    \begin{block}{Avantages}
        \begin{itemize}
            \item Facile d'utilisation
            \item Fiable
        \end{itemize}
    \end{block}
\pause
\begin{block}{Inconvénients}
    \begin{itemize}
        \item Sécurité de la plate-forme.
        \item L'utilisateur cède la gestion de ses données.
        \item Fonctionnement interne opaque.
    \end{itemize}
    $\Rightarrow$ Confiance utilisateur/plate-forme.
\end{block}
\end{frame}

\begin{frame}{Fonctionnement opaque}
    \begin{itemize}
        \item Pas ou peu de documentation technique.
        \item Code inaccessible.
    \end{itemize}
    \pause
    \begin{block}{Ce que l'on sait}
        \begin{itemize}
            \item Fonctionnement en Order Book.
            \item utilisation de bridges inter-blockchains.
        \end{itemize}
    \end{block}
\end{frame}
