\documentclass{beamer}
\usetheme{Boadilla}
\usecolortheme{orchid}

\usepackage[T1]{fontenc}
\usepackage[utf8]{inputenc}
\usepackage[french]{babel}
\usepackage{stackengine}

\addtobeamertemplate{navigation symbols}{}{%
    \usebeamerfont{footline}%
    \usebeamercolor[fg]{footline}%
    \hspace{1em}%
    \insertframenumber/\inserttotalframenumber
}
\usepackage{ulem}
\usepackage{tkz-tab}
\setbeamertemplate{blocks}[rounded]%
[shadow=true]
\AtBeginSection{%
    \begin{frame}
        \tableofcontents[sections=\value{section}]
    \end{frame}
}

\newenvironment*{remerciements}{%
  \renewcommand*{\abstractname}{Remerciements}
  \begin{abstract}
}{
  \end{abstract}
}

\title[Swap Cross-Blockchain]{Echange de jetons inter-blockchains}
\subtitle{Projet de fin d'études: Master 2 Informatique Option Fiabilité et sécurité informatique 2022-2023}
\author[M2 FSI]{VOLPE Dorian, ROTONDO Eloïse, TESTUD Romain,\\DE CAMPOU Louis, JOLY Amaury  \\ \textbf{Encadrants :} TRAVERS Corentin, LABOUREL Arnaud \\[2ex] \includegraphics[scale=0.1]{./img/amu.png}}
\institute[Aix-Marseille Université]{M2 Fiabilité et sécurité informatique}
\date{\today}

\begin{document}

\maketitle

\begin{frame}
  \begin{remerciements}
    Merci à M. TRAVERS Corentin et M. LABOUREL Arnaud pour la proposition de ce sujet et son encadrement.
  \end{remerciements}
\end{frame}

\begin{frame}{Table des matières}
  \tableofcontents
\end{frame}

\section{Introduction}
\subsection{Les Plate-formes d'échanges centralisés}
\subsubsection{Définition}
Nous avons commencé nos recherches en nous intéressant en premier lieu aux moyens d'échanges les plus répandus. 
Cela nous a mené vers les plate-formes d'échanges centralisé. 
Ce sont des plate-formes, pouvant prendre la forme d'applications web, qui permettent aux utilisateurs d'acheter, de vendre ou d'échanger des \gls{actif}s numériques contre d'autres \gls{actif}s numériques ou en monnaies fiduciaires. 
Ces plate-formes peuvent opérer sur des \textit{\gls{blockchain}s} publiques ou être dédiées à une utilisation en interne. 
Elles sont dites centralisées car elles sont gérées par une entreprise ou une organisation hiérarchisée qui contrôle les transactions et les fonds des utilisateurs.
Ces plate-formes sont donc considérées comme des tiers de confiance et agissent en tant qu'intermédiaires entre les acheteurs et les vendeurs en assurant la sécurité, la liquidité et la rapidité des transactions.
C'est la solution la plus utilisée dans le secteur des \gls{actif}s numériques, elles offrent très souvent une certaine variété de services tels que le prêt et/ou le \textit{stacking} \footnote{Stacking : Action de verrouiller des jetons en vue de recevoir des récompenses \cite{defStack}}.
Elles proposent également un large éventail de cryptomonnaies disponibles.

\subsubsection{Inconvénients et risques}
Nous avons pu tout de même relever certains inconvénients et certains risques pour les utilisateurs liés à l'utilisation de ces plate-formes. 
Tout d'abords, les utilisateurs doivent confier leurs fonds et leurs données à un tier en qui ils doivent avoir confiance. 
Cela peut exposer les utilisateurs à de la fraude, du vol ou encore du piratage si les plate-formes présentent des failles de sécurité. \\
Ensuite, ces plate-formes peuvent être victimes de pannes ou de saturation du réseau pouvant entraîner des retards, des pertes de transactions ou encore du déni de service bloquant ainsi l'accès aux \gls{actif}s des utilisateurs. 
Finalement, ces plate-formes sont soumises à la réglementation et à la surveillance des autorités financières, limitant leur accessibilité dans certains pays ou régions. 
Ce point signifie aussi que les \gls{actif}s de l'utilisateurs sont traçables par les autorités. 


\subsubsection{Fonctionnement}
Ces plate-formes fonctionnent sur le principe de l'\textit{order book method} (méthode du carnet d'ordre\cite{orderBook}), une modélisation des ordres d'achats et de vente des jetons.
Un ordre étant une demande d'un utilisateur visant à réaliser une opération à un prix et une quantité donnée. 
Cette méthode comprend deux parties: l'offre et la demande. L'offre regroupe les ordres d'achats émis par des utilisateurs sur la plate-forme et la demande, les ordres de vente.
Lors d'un dépôt, l'utilisateur s'étant au préalable enregistré sur la plate-forme, il va déposer les fonds souhaités dans un porte monnaie. 
La plate-forme va ensuite crée un \textit{IOU}\footnote{I Owe You, c'est la dette de la plate-forme envers l'utilisateur permettant de bloquer la valeur de la monnaie déposée par l'utilisateur\cite{IOU}} 
ce dernier sera échangé contre le crypto-\gls{actif} souhaité lors d'un échange ou d'une vente. \\ 
\begin{figure}[h!]
    \centering
    \includegraphics[scale=0.5]{centralisation/echange.png}
    \label{fig:simplifiedcex}
    \caption{Échange d'un jeton en Euros}
\end{figure}
Dans le cadre des échanges inter-\gls{blockchain}s, les plate-formes d'échanges utilisent des \textit{bridges} reliant les différentes \textit{\gls{blockchain}s}. 
Ces protocoles seront explicités dans la partie suivante.
Cependant, nous n'avons pas pu trouver de plus amples explications quant aux fonctionnements des plate-formes, notamment les protocoles précis utilisés lors des échanges. 
Les documentations disponibles pour les plateformes d'échanges étant à destination des utilisateurs finaux. 




\subsection{Les Blockchain Bridges}
%Auteure : Eloïse Rotondo

\subsubsection{Fonctionnement}

Comme son nom l’indique, un \textit{\gls{blockchain} bridge} également appelé \textit{\gls{cross-chain} bridge} est un protocole reliant deux \textit{\gls{blockchain}s} entre elles de manière unilatérale ou bilatérale dans une optique d’interopérabilité.\\

Dans la but de comprendre la popularité des \textit{bridges} en tant que protocole d’interopéralité, il faut en premier lieu s’intéresser au marché de la cryptomonnaie. Actuellement \gls{Bitcoin} domine en représentant 40,5\% de ce dernier, suivie ensuite par \gls{Ethereum} avec 19,5\%. Cela laisse donc 40\% du marché formé de nombreuses crytomonnaies plus petites et plus indépendantes. C’est donc naturellement, qu’une forte demande de possibilité d’échanges entre les \textit{\gls{blockchain}s} ait vu le jour de la part des utilisateurs ayant plusieurs cryptomonnaies\cite{Ngrave}.\\

Il existe trois différentes manières de déplacer les \gls{actif}s en tant que \textit{bridge}. Tout d’abord, le mécanisme de \textit{Lock and Mint} signifiant Verrouiller et Frapper, les \gls{actif}s se trouvant sur la chaîne de départ sont verrouillés sur celle-ci pour être ensuite créés sur la chaîne destinataire. Un autre mode d'échange est celui du \textit{Burnt and Mint}, ce dernier est très similaire à celui déjà présenté, la seule différence étant que les \gls{actif}s sont directement effacés plutôt que verrouillés. Pour finir, les échanges atomiques entre chaînes (Atomic Swaps) permettent un échange direct en pair-à-pair d'\gls{actif}s entre la chaîne d’origine et la chaîne de destinataire.\cite{EthereumMechanism}

\subsubsection{Mécanisme de vérification}

Comme évoqué précédemment, deux \textit{\gls{blockchain}s} ne peuvent pas communiquer directement entre elles, par conséquent lors de l’utilisation d’un \textit{bridge} les deux chaînes ne se connaissent pas et ont seulement connaissance des évènements se produisant sur leur chaîne respectives. Il est donc nécessaire d’établir une relation de confiance entre les deux chaînes pour qu’elles puissent accepter de communiquer. Pour cela, les \textit{bridges} emploient un mécanisme utilisant des \gls{vérificateur}s. Un \gls{vérificateur} est une entité connectée en tant que \gls{noeud} au réseau de la \textit{blockchain}. Ce dernier agit comme autorité de confiance, vérifiant et validant les transactions sur cette dernière. Un noeud d'une \textit{blockchain} est un ordinateur connecté au réseau de cette dernière. Un \textit{client} est un logiciel permettant de transformer un ordinateur en noeud. \cite{EthereumNodeClient} \\


Il existe un grand nombre de \textit{bridges}, chacun avec leurs propres spécificités mais ils peuvent généralement être séparés en deux catégories les \textit{Trusted \gls{blockchain} Bridge} et les \textit{Trustless \gls{blockchain} Bridge}. \\


Les \textit{Trusted Bridges} sont vérifiés de manière externe car ils utilisent un ensemble de \gls{vérificateur}s tiers pour transmettre des données entre les chaînes. Ils ont pour avantage leur rapidité, leur moindre coût et la facilité d'échange avec tous les types de données acceptés par les \textit{\gls{blockchain}s}. Cependant les \gls{vérificateur}s externes sont moins fiables que ceux de la chaîne.\cite{EthereumBridges}

Les \textit{Trustless Bridges} sont désignés comme \textit{trustless} car ils dépendent des chaînes dont ils font l’intermédiaire pour transférer des données ou des \gls{actif}s. Par conséquent leur niveau de fiabilité est égal à celui des \textit{\gls{blockchain}s} et il n’est pas nécessaire de faire confiance à un ensemble de \gls{vérificateur}s tiers (contrairement aux autres \textit{bridges}). Pour cette raison, ils sont reconnus comme étant plus fiables que les \textit{trusted bridges}.\cite{EthereumBridges}\\


 Les \textit{bridges} peuvent également être distingués en fonction de leur type de vérification. Les plus connus sont la vérification native, externe et locale.\cite{InteroperabilityBhuptani} \\

La vérification native commence par l’utilisation d’un \gls{noeud léger}. \cite{NomadDocsNative} Un noeud léger aussi connu sous le nom de client léger est un logiciel permettant de connecter les noeuds des \textit{\gls{blockchain}s} entre elles. Il est codé comme un \textit{\gls{smart contract}} puis est employé de la chaîne de l'expéditeur vers la machine virtuelle de la chaîne destinataire. Si les \gls{vérificateur}s de données de la chaîne de l'expéditeur agissent de manière correcte alors le noeud léger est vu comme véridique par la chaîne réceptionnant les \gls{actif}s ou données et peut être utilisé de manière bilatérale.
 Un avantage de cette solution est qu’elle est reconnue comme étant celle reposant le moins sur la confiance parmi celles existantes car les chaînes ne se fient qu’à leurs propres \gls{vérificateur}s pour effectuer le \textit{bridge}. Un autre bénéfice de ce mécanisme est le fait qu’il n’utilise pas de \gls{vérificateur}s tiers entre les deux \textit{\gls{blockchain}s} et donc la sécuté du réseau dépend des \textit{\gls{blockchain}s} elles-même (ce qui est avantageux car elles sont robutes et préparées aux attaques comme la chaîne d’Ethereum par exemple).
 Un désavantage de cette méthode est que le noeud léger doit être adapté aux consensus des chaînes auquelles il est attaché ce qui le rend inutilisable avec des chaînes différentes. Le noeud léger nécessite également de la maintenance en cas de changement des règles consensus (utilisées pour valider les transactions). Un autre inconvénient découlant du fait que le noeud léger est programmé de manière spécifique est que ce dernier n’est donc pas réutilisable. \\

\pagebreak

La vérification externe consiste en un ensemble de \gls{vérificateur}s n’appartenant pas aux \textit{\gls{blockchain}s} relayant les données entre les deux extrémités du \textit{bridge}. Pour se faire, un certains nombre de \gls{vérificateur}s doivent signer un message provenant de la chaîne d’envoi pour que le chaîne destinataire le reconnaisse comme valide. Par exemple, pour le \textit{bridge} \gls{Wormhole} 13 \gls{vérificateur}s sur 19 doivent avoir signé\cite{NomadDocsExternal}. Ce concept est une primitive cryptographique (algorithme cryptographique de bas niveau servant de base à un système de sécurité informatique) nommée le système de signature à seuil (désignée par TSS pour \textit{Threshold Signature Scheme})\cite{BinanceTSS}. 
Contrairement à la vérifications native, les \textit{bridges} vérifiés de manière externe sont faciles à développer, peuvent être réutilisés sans problèmes et leur maintenance coûte peu. Le désavantage conséquent de cette méthode est que la sécurité dépend des \gls{vérificateur}s tiers du pont ce qui peut fragiliser le système car ils sont généralement moins sécurisés que ceux des \textit{\gls{blockchain}s}. \\


\begin{figure}[h!]
    \centering
\stackunder{
\includegraphics[scale=0.60]{centralisation/imagesBridges/LightClient.png}}
    {\scriptsize
            Source: \url{https://docs.nomad.xyz/the-nomad-protocol/verification-mechanisms/native-verification}}
    \caption{Mécanisme utilisant un noeud léger.}
    \label{fig:LightClient}
\end{figure}

\pagebreak

La vérification externe consiste en un ensemble de \gls{vérificateur}s n’appartenant pas aux \textit{\gls{blockchain}s} relayant les données entre les deux extrémités du \textit{bridge}. Pour se faire, un certains nombre de \gls{vérificateur}s doivent signer un message provenant de la chaîne d’envoi pour que le chaîne destinataire le reconnaisse comme valide. Par exemple, pour le \textit{bridge} Wormhole 13 \gls{vérificateur}s sur 19 doivent avoir signé\cite{NomadDocsExternal}. Ce concept est une primitive cryptographique (algorithme cryptographique de bas niveau servant de base à un système de sécurité informatique) nommée le système de signature à seuil (désignée par TSS pour \textit{Threshold Signature Scheme})\cite{BinanceTSS}. 
Contrairement à la vérifications native, les \textit{bridges} vérifiés de manière externe sont faciles à développer, peuvent être réutilisés sans problèmes et leur maintenance coûte peu. Le désavantage conséquent de cette méthode est que le bon fonctionnement du système dépend des \gls{vérificateur}s tiers ce qui peut le fragiliser car ils sont généralement moins fiables que ceux des \textit{\gls{blockchain}s}. \\

Il est intéressant de noter que les \textit{\gls{blockchain}s} ont également leur propre ensemble de \gls{vérificateur}s sous la forme de \gls{vérificateur} de données. Ces derniers sont utilisés lors de la vérification locale. Lors de la vérification locale, les chaînes se vérifient entre elles en utilisant un \gls{vérificateur} en tant que représentant.

\begin{figure}[h]
    \centering
\includegraphics[scale=0.60]{centralisation/imagesBridges/DiagrammeResumeVerif.png}
\caption{Résumé des types de vérification. (locale, native et externe)}
\label{fig:LocaleVerif}
\end{figure}

\subsubsection{Les risques liés aux Bridges}

La popularité des \textit{\gls{blockchain} Bridges} pour les échanges centralisés ne cesse d’augmenter au fils du temps mais il est important de comprendre que comme tout outil, ces derniers ne sont pas sans risques. \\

Les \textit{bridges trustless} utilisent des \textit{\gls{smart contract}s} lors du processus d’échange dans le but de le rendre autonome afin de ne pas utiliser une entité centrale entre les deux \gls{blockchain}s. Cependant ces \textit{bridges} sont néanmoins centralisés car ils utilisent les \gls{vérificateur}s pour obtenir un consensus lors des transactions.
Un \textit{\gls{smart contract}} étant un script écrit par un développeur, il est possible que certaines erreurs puissent s’être glissées dans le code par inadvertance ou bien qu’il existe des failles dans le programme permettant aux attaquants de le détourner pour un profit personnel. 
Pour minimiser ce type de risques, il est recommandé d’effectuer des audits sur les \textit{bridges.} \\

Une faiblesse spécifique des \textit{bridges trusted} repose sur le fait que les utilisateurs doivent léguer le contrôle de leurs \gls{actif}s et faire confiance aux \gls{vérificateur}s externes aux \gls{blockchain}s. Sauf que dans certains cas, ces derniers peuvent coopérer pour tromper les utilisateurs en récupérant leurs \gls{actif}s puis en disparaissant comme dans les \textit{rug pull}\cite{EthereumRisks}. Ce modèle d’escroquerie peut être scindé en deux catégorie : les \textit{hard rug pull} et les \textit{soft rug pull}\cite{Hacken}. Le premier cas est basé sur un piège présent dans le code d’un \textit{\gls{smart contract}} empêchant les utilisateurs d’utiliser ou revendre les \gls{actif}s frappés, seul le fraudeur en a le droit. Il peut donc en toute tranquillité revendre les \gls{actif}s et récupérer l’argent. En revanche, pour les \textit{soft rug pull}, les utilisateurs ne sont pas coincés avec des \gls{actif}s verrouillés mais les fraudeurs utilisent des techniques psychologiques. En effet, les escrocs rendent attirant leur projet pour que les clients investissent et hésitent à se retirer par peur de perdre leur investissent (souvent de taille conséquent) puis les créateurs de la fraude disparaissent avec leurs \gls{actif}s.\\

Comme vu dans la section présentant les différentes méthodes d’échange des \textit{bridges}, ces derniers frappent les \gls{actif}s désirés sur la chaîne destinataire. Certains attaquants peuvent profiter de ce mécanisme de frappe pour effectuer ce qu’on appelle une \textit{Infinite Mint Attack}.\cite{ChainLinkRisks} Cette attaque peut se résumer à un \textit{hacker} générant un nombre élevé d’\gls{actif}s en utilisant une faille d’un \textit{bridge} sans verrouiller ou brûler d’\gls{actif}s sur sa \textit{\gls{blockchain}}. Suite à cela, l’individu réintroduit ces \gls{actif}s sur le marché ce qui fait violemment baisser leur coût ce qui engendre un risque financier systémique.\\

Les \textit{\gls{blockchain} Bridges} sont devenus un outil indispensable des échanges centralisés très rapidement, mais il ne faut pas oublier que ces protocoles sont relativement récents. Créés par de petites \gls{blockchain}s comme Syscoin et NEAR Protocol dans le but de rentre leurs chaînes interopérables avec les applications décentralisées d’\gls{Ethereum}, les premiers bridges datent de 2020\cite{Bitstamp}. Par conséquent, nous ne connaissons pas encore le comportement des \textit{bridges} lorsqu’ils font face à des scénarios sortants de la norme comme des attaques réseaux, un retour en arrière sur les transactions d’une \gls{blockchain} (souvent désigné par le terme \textit{rollback}) ou bien pendant une congestion du réseau. Ces zones d’incertitudes peuvent donc être une source de risques. \\

\begin{figure}[h!]
    \centering
\includegraphics[scale=0.30]{centralisation/imagesBridges/GraphLossesBridges.png}
    {\scriptsize
            Source: \url{https://www.treehouse.finance/insights/blockchain-and-interoperability-globalization-3-0}}
    \caption{Pertes en millions de dollars des bridges les plus connus.}
    \label{fig:GraphBridges}
\end{figure}

\subsection{Le trilemme de l’interopérabilité}

Malgré l’existence de plus d’une centaine de \textit{bridges} différents, les développeurs et les utilisateurs voulant utiliser un \textit{bridge} doivent faire des concessions lors de leur choix vis-à-vis des trois notions de \textit{trustless}, d’extensibilité (\textit{extensible}) et de généralisation (\textit{generalizable}).\\

Le mot trustless peut être traduit par «sans confiance». Si un \textit{bridge} est caractérisé comme \textit{trustless}, cela signifie que celui-ci possède un niveau équivalent à celui d’une ou des chaînes sous-jacentes, il est donc pas nécessaire de faire confiance à une entité externe aux \textit{blockchains}.  La notion  d’extensibilité signifie que le \textit{bridge} est compatible un grand nombre de chaînes.
Un \textit{bridge} respecte la notion de généralisation s’il est capable d'échanger n’importe quel type de données accepté par les deux chaînes.\\

Pour illustrer ces termes, il est possible de les appliquer aux types de vérification appartenant aux \textit{bridges}. La vérification locale respecte les notions d’extensibilité et de \textit{trustless} puisque qu'elle est applicable sur tous les \textit{bridges} peu importe les chaînes reliées et le niveau de fiabilité dépend de la chaîne la plus faible.
La vérification native ne respecte pas la notion d’extensibilité car le \textit{bridge} n’est pas réutilisable. Néanmoins elle respecte la notion de généralisation parce qu’elle est codée de manière spécifique aux \textit{\gls{blockchain}s} reliées au \textit{bridge}. Le critère basé sur la notion \textit{trustless} est également rempli étant donné que le niveau de fiabilité dépend des \gls{vérificateur}s des chaînes.
La vérification externe ne respecte pas la notion de \textit{trustless} cependant elle est fortement extensible et générale vis-à-vis des données. \cite{Ngrave}

 Suite au paragraphe précédent, il est possible de constater que les \textit{bridges} interopérables ne respectent que deux des trois notions énoncées. Ce problème est connu sous le nom de trilemme de l’interopérabilité. 

\subsubsection{Une solution optimiste}

Une solution proposée pour résoudre ce trilemme est un \textit{optimistic bridge}(\textit{bridge} optimiste) nommé vis-à-vis de sa vérification portant le même nom\cite{OptimisticBhuptani}. En effet, contrairement aux \textit{bridges} vérifiés de manière native, locale ou externe, la vérification optimiste dépend de l’utilisation d’une latence lors de la confirmation du transfert des \gls{actif}s entre les \textit{\gls{blockchain}s}. Cela priorise donc la sécurité au détriment de la vivacité, puisque les transactions sont par conséquent plus lentes mais sécurisées car le \textit{bridge} est \textit{trustless}. \\

Voici plus en détails le déroulement du processus de vérification optimiste d’un \textit{bridge}.
Ceci commence par l’envoi d'une demande de transaction de la part d’un utilisateur ou d’une application décentralisée vers la \textit{\gls{blockchain}} native par le biais d’une fonction contrat. Cette demande est ensuite acceptée par un validateur (une entité équivalente à un \gls{vérificateur}, la seule différence étant leur rôle) puis inscrite sur un \textit{block} de la chaîne.\cite{NomadDocsVerification}
Ensuite un \gls{vérificateur} a pour rôle de vérifier la transaction. Pour cela, le \gls{vérificateur} signe le haché des données envoyées précédemment. 
Suite à cela, n’importe quel système de relais peut lire le haché signé sur la chaîne originelle et l’inscrire sur une ou plusieurs chaînes destinataire. Les validateurs de la chaîne destinataire valide et l'inscrivent sur leur chaîne. Cette action déclenche alors une latence de trente minutes pendant laquelle un observateur peut signaler et prouver une fraude effectuée par la chaîne native ce qui déconnectera la communication avec la chaîne destinataire. 

Deux scénarios sont alors possibles. Premier cas, si aucun observateur ne se manifeste, les données de la transaction sont finalisées puis traitées par la chaîne destinataire. Les validateurs sont récompensés pour leur travail avec une partie des frais de transaction. \cite{Fees} Deuxième cas, un observateur prouve une fraude pendant les trente minutes accordées. Le \gls{vérificateur} ayant fraudé est pénalisé par la perte de sa récompense (qui sera obtenu par l'observateur) et son exclusion du réseau.\cite{EthereumSlashing}

\subsubsection{Possibles faiblesses de l’optimisme et leurs solutions}

Le bon fonctionnement des \textit{bridges} optimistes dépend des chaînes auquelles il est rattaché donc tant que ces dernières sont protégées correctement la seule conséquence d’une faille du \textit{bridge} est l’arrêt système plutôt qu’une perte de fonds comme cela peut être le cas avec les autres types de \textit{bridge}.

Les deux acteurs principaux ayant les moyens de nuire au bon fonctionnement du \textit{bridge} ainsi que de la transaction est l’agent \gls{vérificateur} ainsi que l’observateur car ces deux rôles ont de l’influence sur cette dernière. \\

Le premier cas impliquant le \gls{vérificateur} se nommant \textit{Updater Fraud} (fraude du \gls{vérificateur})  fut déjà mentionné lors de la présentation du fonctionnement du \textit{bridge} optimiste. Ce dernier repose sur le fait que toute transaction doit passer par le \gls{vérificateur} et que par conséquent toute fraude est originaire de ce dernier. Sinon si cela venait d’un autre participant, le \gls{vérificateur} n’aurait alors pas accepté la transaction. C’est pourquoi, lors de l’intervention d’un observateur prouvant une fraude, le \gls{vérificateur} est sanctionné par le retrait sur son solde d'un montant équivalent à la récompense promise et par son exclusion du réseau de la \textit{blockchain}. \\

La seconde faiblesse liée aux \gls{vérificateur}s est un \textit{Updater DoS} ou déni de services de la part du \gls{vérificateur}. En effet, il est possible que le processus soit interrompu si un validateur arrête de signer empêchant l'échange inter-chaînes de se produire.
Une solution a été implémentée pour palier à cela comme la mise en place d’un système de substitution avec la présence de plusieurs \gls{vérificateur}s sur une même chaîne afin de pouvoir prendre le relai en cas de manque de réponse de la part de celui étant rattaché au transfert.  Pour éviter que ce scénario se produise fréquemment le \gls{vérificateur} ayant manqué son tour lors de la signature (que cela soit accidentel ou voulu) est pénalisé de la même manière que le cas précédent. \\

Maintenant que les possibles obstacles au bon fonctionnement du \textit{bridge} liés aux \gls{vérificateur}s ont été mis en lumière, il est également possible que l’observateur ait un comportement malveillant.  Effectivement, malgré l’absence de tromperie (puisque le \gls{vérificateur} remplit son rôle), l’observateur peut abuser du mécanisme de déclaration de fraude pour impacter le bon déroulement du procédé. \\

La faculté de l’observateur à pouvoir couper la connexion s’il conteste la transaction lui permet d’effectuer un déni de service appelé \textit{Watcher DoS}. C’est pourquoi il lui ait possible de fermer définitivement la connexion d’une transaction si ce dernier continue sans cesse de couper le processus sans raison valable. Heureusement, la fermeture ne concerne que la connexion et n’impacte en aucun cas le système du \textit{bridge}. Cependant cette attaque semble irrationnelle en terme de ressources et de temps car l’observateur effectuant le déni de service ne gagne rien financièrement contrairement au processus habituel. En effet, si un observateur prouve une fraude correctement, ce dernier peut récupérer la récompense du \gls{vérificateur}. Mais ici puisqu’aucune fraude n’est prouvée les données se trouvant sur la chaîne d’origine sont conservées et sécurisés. Cela cause seulement une perte de temps pour l’utilisateur ou l’application décentralisée voulant effectuer l'échange d’une \textit{\gls{blockchain}} à une autre.

Une réponse à ce problème actuellement mise en œuvre par le \textit{bridge} de \gls{Nomad} est la présence d’un groupe restreint d’observateurs autorisés à contester, de cette manière il est facile de connaître les observateurs malveillants. Chaque observateur possède une clé permettant de signer une attestation confirmant la présence d’une fraude dans la transaction, chaque \textit{bridge} stocke un ensemble contenant les adresses des attestations appartenant aux observateurs autorisés. Si l’attestation reçue par le \textit{bridge} est présente dans l’ensemble alors la connexion est rompue\cite{NomadDocsWatcher}.
Sur le long terme, une proposition consistant à la mise en place de frais si l’on souhaite contester est en train d’être étudiée. Le montant doit répondre à deux contraintes: ce dernier doit être assez haut pour dissuader les observateurs malhonnêtes mais assez bas pour que ceux ayant réellement l’envie de prouver de manière valide une fraude existante puissent le faire. Dans la continuité de cette solution, il serait également possible de récupérer la signature de la déconnexion générée par l’observateur sur la chaîne originale et de le pénaliser en lui retirant les frais qu’il a payé tel une garantie\cite{OptimisticBhuptani}.



\subsection{Wormhole}
En 2017, une cryptomonnaie adossée à la \textit{blockchain} Solana a émergée avec des caractéristiques
similaires à Ethereum : \textit{blockchain} publique, \textit{smart contracts}.\\
Solana est devenue de facto une \textit{blockchain} concurrente à Ethereum et est aujourd'hui 
la onzième \textit{blockchain} en terme de capitalisation selon l'aggrégateur de marché Coinmarketcap.\\
Un besoin d'échanger des actifs entre les \textit{blockchains} Ethereum et Solana est apparu, 
d'où l'introduction en 2020 de la première version de Wormhole.
Initialement, Wormhole v1 a été concu comme un \textit{bridge} entre Ethereum et Solana.
Depuis, Wormhole s'est développé au-delà de Solana avec le lancement d'une deuxième version en 2021 
en tant que protocole générique de passage de messages.\\
À l'écriture de ce rapport, 22 \cite{wormholeNetwork} \textit{blockchains} sont compatibles avec Wormhole 
dont : BNBChain, Ethereum, Moonbeam, Polygon, Solana...\\
Le protocole émet un message à partir d'une \textit{blockchain} source qui est validé par un réseau de 
gardiens.\\ 
Le message est ensuite envoyé à la \textit{blockchain} cible pour être traité.

\subsubsection{VAA (\textit{Verified action approval})}

Lorsqu'un \textit{smart contract} envoie un message \textit{crosschain} comme un verrouillage
de jetons sur une \textit{blockchain} source et une demande de frappe de jetons sur une 
\textit{blockchain} cible, celui-ci interargit avec un \textit{core contract} \cite{wormholeCoreContract}.
Un \textit{core contract} est déployé sur toutes les \textit{blockchains} compatibles avec le protocole 
Wormhole. Tout \textit{core contract} est observé par le réseau de gardiens.
Un message Wormhole est émis grâce à la fonction \textit{publishMessage()} prenant en entrée le \textit{payload}.
La sortie de cette fonction est un \textit{sequence number}, un numéro d'index unique pour le message.
Combiné à l'adresse du contrat de l'émetteur et à l'identifiant de la chaîne de l'émetteur, le message 
correspondant peut être récupéré auprès d'un nœud du réseau de gardiens.\\
Un message Wormhole est vérifié grâce à la fonction \textit{parseAndVerifyVAA()} prenant en entrée le message.
Selon la validité de l'entrée, la fonction retourne en sortie le \textit{payload} ou une exception.
\newpage

VAA \cite{wormholeVAA} est la primitive de messagerie de base de Wormhole. Un VAA contient une en-tête 
ainsi qu'un \textit{body}. L'en-tête contient l'index des gardiens ayant signés le message et la collection des signatures.
L'en-tête permet au \textit{core contract} de vérifier l'authenticité du VAA.
Quant au \textit{body}, il contient des informations comme le numéro d'identification de la chaîne 
Wormhole du contrat émetteur, l'adresse du contrat émetteur, le \textit{sequence number} 
et le \textit{payload}.\\ 5 \textit{payloads} peuvent être utilisés dont \textit{Transfer} et 
\textit{AssetMeta}, attestant les méta-données du jeton.\\
Le \textit{payload AssetMeta} est obligatoire avant un premier transfert.
En effet, le \textit{payload Transfer} n'informe pas la chaîne B des meta-données du jeton verrouillé.
En l'absence de connaissance de ces informations, il n'est pas possible pour la \textit{blockchain} B 
de frapper la quantité correcte de jetons.\\
Si l'on souhaite ensuite transférer des jetons depuis une \textit{blockchain} A vers une 
\textit{blockchain} B, il faut verrouiller les jetons sur A et les frapper sur B.
D'où l'utilisation du \textit{payload Transfer} contenant des informations comme la 
quantité de jetons transférés, l'adresse de la chaîne d'origine et de destination, 
le numéro d'identification de la chaîne d'origine et de destination..
Une preuve doit être fournie que les jetons sur A sont verrouillés avant que la frappe puisse 
avoir lieu sur B. La signature des gardiens sur le VAA correspondant est la preuve apportée à 
la \textit{blockchain} B que le verrouillage a bien été effectué et que la frappe de jetons sur 
B est légitime.

\subsubsection{Gardiens}

Un gardien \cite{wormholeGuardian} est une autorité de confiance qui a comme de rôle valider 
(par une signature) le \textit{payload} contenu dans un VAA.
Comme évoqué précédemment, le réseau de gardiens observe tous les messages \textit{crosschain} via la 
surveillance des \textit{core contracts}.
Le réseau de gardiens est composé de 19 gardiens à parts égales sans chef (\textit{leaderless}).
Il est conçu pour servir d'oracle à Wormhole et est l'élement le plus critique de l'écosystème.
Si une majorité de deux tiers ou plus des gardiens signent le même VAA, alors le consensus est atteint : 
le VAA est automatiquement considéré valide par  tous les contrats Wormhole sur toutes les 
\textit{blockchains} et le \textit{payload} est actionné. 
Chaque gardien utilise un algorithme de signature à courbe elliptique : ECSDA pour 
\textit{Elliptic Curve Signature Digital Algorithm}.
Plus précisément, chaque gardien se réfère à «secp256k1» comme paramètres de la courbe elliptique, 
aussi utilisé par les \textit{blockchains} Bitcoin et Ethereum.\\
Le modèle de consensus utilisé est une \textit{Proof of Authority} (PoA) avec un système de 
\textit{multisignature} M/N \cite{wormholeChainswap}, c'est à dire que M clefs parmi N sont nécessaires 
pour signer un VAA. Ce modèle permet un traitement rapide des transactions et une dispense de participation monétaire, par rapport à la preuve de travail (PoW) et la preuve 
de participation (PoS). Cependant, il présente également des désavantages : le système est par 
\textit{design} centralisé et dépend d'un petit groupe de nœuds pouvant créer un point de 
défaillance unique par l'utilisation commune d'une fonction vulnérable. Il est questionnable de restaurer des tiers de confiance dans le cadre d'un système 
devenu populaire grâce à l'absence de tels autorités. Wormhole justifie la décentralisation de leur 
système \cite{wormholeGuardian} par la présence de plusieurs parties (et non d'un seul) dans le contrôle du réseau.
Selon notre analyse, la décentralisation résulte de l'absence d'un ou plusieurs tier(s) de confiance lorsque deux parties 
souhaitent réaliser une transaction.
\newpage

\subsubsection{Relais}

Un relai \cite{wormholeRelayer} est un processus qui délivre un ou plusieurs VAA(s) à une destination.
Les relais ne sont ni de confiance, ni privilégiés, ils écoutent directement le réseau de gardiens 
via l'intermédiaire d'un processus espion. Ces relais ne peuvent pas compromettre l'intégrité d'un VAA 
car une altération serait détectée lors du processus de vérification des signatures. Cependant, il n'est 
pas assuré qu'un relai transmette un VAA à destination, d'où une perte de disponibilité. Il est conseillé
d'héberger soi-même ces relais pour supporter son application.

\begin{figure}[h!]
    \centering
    \includegraphics[scale=0.5]{centralisation/uml_design_v2.png}
    \label{fig:wormholeDesign}
    \caption{Architecture Wormhole \cite{wormholeArch}}
  \end{figure}

% @startuml
% rectangle r1 as "Source Token Bridge 
% Relayer Contract"
% rectangle r2 as "Source Token 
% Bridge"
% rectangle r3 as "Source Core 
% Contract"
% storage "Guardians" as r4
% hexagon r5 as "Off-chain 
% Message Relayer" 
% rectangle r6 as "Target Token Bridge 
% Relayer Contract"
% rectangle r7 as "Target Token 
% Bridge"
% rectangle r8 as "Target  Core 
% Contract"

% r1 -> r2 : Transfer()
% r2 -> r3 : publishMessage()
% r3 -do-> r4 : Guardien reads message
% r4 -left-> r5 : Signed message

% r5 -do-> r6 : Signed message
% r6 -> r5 : Relayer fee

% r6 -> r7 : Signed message
% r7 -> r6 : Wrapped token

% r7 -> r8 : parseAndVerifyVAA()
% @enduml




\subsection{Analyse d'attaques}
% Auteur Romain TESTUD
\subsubsection{Mise en contexte}
Les \textit{\gls{blockchain}s} et leurs protocoles d'échanges ne sont pas exemptes d'attaques informatiques ou bien de défaillances.
Ces attaques peuvent cibler des portefeuilles (attaques sur des \textit{hot wallets}\footnote{portefeuille de cryptomonnaies en ligne, à différencier des \textit{Cold Wallets}, des portefeuilles hors lignes}) ou encore des \textit{bridges}. 
Ce sont ces dernières qui nous ont intéressées dans le cadre de ce projet de recherche sur les échanges inter-\gls{blockchain}s. 
Les bridges, comme explicité dans la partie dédiée du rapport, sont des protocoles permettant la circulation de données entre deux \textit{\gls{blockchain}s} différentes.\\
Nous avons, au cours de nos recherches, trouvés de nombreux cas d'attaques sur des protocoles d'échanges inter-\gls{blockchain}s. 
De manière à illustrer les types d'attaques possibles et les points critiques de ces systèmes nous allons décrire deux attaques parmi les plus importantes : \textit{Wormhole} et \textit{Nomad}.

\subsubsection{Le cas Wormhole}
Nous vous avons présenté le protocole \textit{Wormhole} dans la partie précédente. 
Le 2 Février 2022, une attaque exploite une erreur d'implémentation dans une \textit{\gls{dApp}} sur la chaîne Solana \cite{SolMed} \cite{SolRekt}. 
Pour se faire l'attaquant à réussi à contourner la vérification des signatures des gardiens en exploitant
une correction de bug ayant été publié sur le code source du projet mais n'étant pas encore effective en production.
Ainsi il à réussi à récupérer l'équivalent de 120 000 \textit{ETH} en \textit{whETH} (\textit{Wormhole ETH}). 
Lors d'un transfert de jetons d'une chaîne à une autre, plusieurs étapes sont réalisées par différentes fonctions.
Après la formulation de la transaction, une fonction va se charger de récupérer les signatures des gardiens dans un \textit{SignatureSet}\footnote{Ensemble de signatures de gardiens}, ces dernières sont ensuite vérifiées. 
Pour cela, une fonction nommée \texttt{verify\_signature} va appeler un programme de vérification de Solana permettant l'analyse du \textit{SignatureSet}. 
L'appel à ce programme se fait de la manière suivante, en utilisant le nom \texttt{sysvarinstruction} \cite{SolGitError} dans la transaction. 
Dès lors que les signatures sont validées, un \textit{VAA} peut être émis et transmis vers la \textit{\gls{blockchain}} souhaitée. \\
La transaction de l'attaquant étant frauduleuse, il n'aurait donc pas pu obtenir de signatures des gardiens. 
Pour contourner cette étape de récupération des signatures la transaction de l'attaquant était dotée d'un \textit{SignatureSet} correspondant à une transaction antérieure. 
Seulement, n'étant pas pour la bonne opération cet ensemble ne peut pas être approuvé par \texttt{verify\_signature}. 
C'est ici que l'attaquant à utilisé un défaut d'implémentation pour valider son \textit{SignatureSet}. 
Comme décrit précédemment, la fonction \texttt{verify\_signature} appelle un programme pour effectuer la vérification des signatures. 
Cependant il n'y à pas de vérification faites sur le programme appelé, l'attaquant à pu donc utiliser une adresse différente lui permettant de valider sa transaction. 
Avec le  \texttt{SignatureSet} ainsi validé, l'attaquant a pu générer un \textit{VAA} valide et pu déclencher une frappe de jeton vers son propre compte sans en avoir déposé au préalable. 
La correction de cette faille était contenue dans la mise à jour évoquée en début de paragraphe\cite{SolGitFixed}, permettant la vérification du programme appelé pour la vérification. 

\subsubsection{Le cas Nomad}
Nomad est un protocole d'interopérabilité entre chaînes permettant de passer des \gls{actif}s entre deux \textit{\gls{blockchain}s} différentes. 
Pour fonctionner, ce protocole fait appel à des applications décentralisées opérant sur les chaînes du réseau. 
Une première \textit{\gls{dApp}} appelée \textit{réplica} est déployée sur les \textit{\gls{blockchain}s} recevant les messages, elle fait office de "boite de réception". 
Une seconde \textit{\gls{dApp}} appelée \textit{home} est déployé sur les \textit{\gls{blockchain}s} émettrices de message. \\
Le 1\textsuperscript{er} août 2022 une attaque exploitant une erreur d'implémentation sur l'application \textit{Réplica} a engendré une perte de 190 millions de dollars en liquidité \cite{NomadMedium} \cite{NomadRekt}.
Cette attaque s'est déroulée après le déploiement d'une mise à jour, un moyen de contourner la vérification des signatures du message étant apparu. 
En analysant l'application \textit{Réplica} après la mise à jour, nous pouvons voir que lors d'une initialisation, la racine des messages, appelée \texttt{\_commitedRoot}, est initialisée à $0$, ce signifiant que le message n'a pas encore été validé. 
\begin{lstlisting}[caption={Fonction \textit{initialize} de \textit{Réplica} contenant une erreur \cite{NomadGitError}}]
    function initialize(
        uint32 _remoteDomain,
        address _updater,
        bytes32 _committedRoot,
        uint256 _optimisticSeconds
    ) public initializer {
        __NomadBase_initialize(_updater);
        // set storage variables
        entered = 1;
        remoteDomain = _remoteDomain;
        committedRoot = _committedRoot;
        // pre-approve the committed root.
        confirmAt[_committedRoot] = 1;
        _setOptimisticTimeout(_optimisticSeconds);
    }
\end{lstlisting}

Dans les lignes précédentes nous observons cette affectation : \texttt{confirmAt[\_commitedRoot] = 1}, le rôle de cette ligne est de pré-approuver la racine d'un message. 
Cette fonction est utilisée pour approuver le premier message lors du déploiement du contrat sur une \textit{\gls{blockchain}}. 
Or ici, la valeur de la racine à été initialisée a $0$, donc cette racine devient une racine valide pour la fonction de vérification des messages. 
Seulement comme nous l'avons énoncé précédemment, $0$ est la valeur par défaut pour un message n'ayant pas encore été vérifié. 
Ainsi, lors de l'émission d'un message par la fonction \texttt{process}, tout message non vérifié sera envoyé. 
Cette erreur d'implémentation a permis à des pirates d'effectuer plusieurs transactions frauduleuses et de retirer l'équivalent de 190 Millions de dollars dans la réserve de liquidité du bridge de Nomad. 
Le contrat à été corrigé, dans une mise en ligne datant du 3 Septembre 2022, tel que la racine $0$ n'est plus pré-approuvée. 

\begin{lstlisting}[caption={Fonction corrigée de l'application \textit{Réplica} \cite{NomadGitFixed}}]
    function initialize(
        uint32 _remoteDomain,
        address _updater,
        bytes32 _committedRoot,
        uint256 _optimisticSeconds
    ) public initializer {
        __NomadBase_initialize(_updater);
        // set storage variables
        entered = 1;
        remoteDomain = _remoteDomain;
        committedRoot = _committedRoot;
        // pre-approve the committed root.
        if (_committedRoot != bytes32(0)) confirmAt[_committedRoot] = 1;
        _setOptimisticTimeout(_optimisticSeconds);
    }
\end{lstlisting}


%\subsubsection{DeFi hacklabs}
%Lors de nos recherches sur des attaques sur des protocoles inter-\gls{blockchain}s, nous avons découvert \textit{Web3sec}, un groupe de recherche centré sur la sécurité du web3. 
%Le groupe met à disposition des ressources indexés sur une page notion (en annexe) :
%\begin{itemize}
%    \item Plusieurs dépots \textit{Github} pour étudier les attaques et apprendre les vulnérabilités sur ces types de programmes.
%    \item \textit{DeFi Hacks Analysis - Root Cause} : Une base de données d'analyses d'attaques sur des solutions et organismes traitant sur des \gls{blockchain}s, les analyses sont sourcées et redirigent vers le dépot \textit{GitHub} de reproduction des attaques.
%    \item Un ensemble d'outils utilitaires tels que des outils de debug de transaction, des dashboards ou encore des newletters.
%\end{itemize}
%Ces outils nous ont permis d'explorer de nombreuses sources concernant les attaques sur les protocoles inter\gls{blockchain}s.

\subsubsection{Contre-mesures et solutions envisageables}
Comme nous avons pu le voir, de nombreux cas d'attaques sont observables sur des protocoles d'échanges centralisés. 
Dans la plupart des cas, elles résultent de problèmes d'implémentations et autres oublis dans les codes sources des protocoles utilisés. 
Cela peut être expliqué par le fait que la \textit{\gls{blockchain}} est un domaine qui évolue très vite et chaque innovation technique peut rapporter des parts de marché importantes au premier arrivé. 
De plus, de nombreux acteurs se spécialisent dans ce domaine sans nécessairement avoir une grande culture de la cybersécurité.
Il se peut donc que des erreurs d'implémentations paraissant évidentes ne soient pas relevés lors de la mise en production. \\ 
Des moyens de limiter le plus possible l'apparition de tels événements sont néanmoins possibles. 
Tout d'abord des standards de sécurité pourraient être mis en place afin de déterminer un socle minimal à atteindre. 
Des audits et des analyses de sécurité peuvent être mis en place pendant la production ou après la publication des protocoles, en respectant un cycle de développement "classique".
%Nous pouvons aussi nous questionner quant à l'implication du tier de confiance dans l'apparition de ces failles. 
%En effet, les attaques les plus importantes ont comme point d'entrée une faille dans la structure de l'intermédiaire. 
%Cette menace constante d'attaques a mené à des recherches visant à soutirer ce tiers des échanges inter-\gls{blockchain}. Ce menant aux échanges décentralisés.


\subsection{Les limites du centralisé}
% Auteur : Romain TESTUD
\begin{frame}{Les limites du centralisé}
    \begin{itemize}
        \item Manque de transparence.
        \item Manque de maturité en matière de sécurité.
        \item Questionnement sur le tiers de confiance
    \end{itemize}
    \pause
    \begin{block}{Le tiers de confiance}
        \begin{itemize}
            \item Source d'attaques.
            \item Point critique des CEX.
        \end{itemize}
        $\rightarrow$ Volonté de se passer d'un tiers de confiance.
    \end{block}
\end{frame}
\section{Centralisation}
\subsection{Les Plate-formes d'échanges centralisés}
\subsubsection{Définition}
Nous avons commencé nos recherches en nous intéressant en premier lieu aux moyens d'échanges les plus répandus. 
Cela nous a mené vers les plate-formes d'échanges centralisé. 
Ce sont des plate-formes, pouvant prendre la forme d'applications web, qui permettent aux utilisateurs d'acheter, de vendre ou d'échanger des \gls{actif}s numériques contre d'autres \gls{actif}s numériques ou en monnaies fiduciaires. 
Ces plate-formes peuvent opérer sur des \textit{\gls{blockchain}s} publiques ou être dédiées à une utilisation en interne. 
Elles sont dites centralisées car elles sont gérées par une entreprise ou une organisation hiérarchisée qui contrôle les transactions et les fonds des utilisateurs.
Ces plate-formes sont donc considérées comme des tiers de confiance et agissent en tant qu'intermédiaires entre les acheteurs et les vendeurs en assurant la sécurité, la liquidité et la rapidité des transactions.
C'est la solution la plus utilisée dans le secteur des \gls{actif}s numériques, elles offrent très souvent une certaine variété de services tels que le prêt et/ou le \textit{stacking} \footnote{Stacking : Action de verrouiller des jetons en vue de recevoir des récompenses \cite{defStack}}.
Elles proposent également un large éventail de cryptomonnaies disponibles.

\subsubsection{Inconvénients et risques}
Nous avons pu tout de même relever certains inconvénients et certains risques pour les utilisateurs liés à l'utilisation de ces plate-formes. 
Tout d'abords, les utilisateurs doivent confier leurs fonds et leurs données à un tier en qui ils doivent avoir confiance. 
Cela peut exposer les utilisateurs à de la fraude, du vol ou encore du piratage si les plate-formes présentent des failles de sécurité. \\
Ensuite, ces plate-formes peuvent être victimes de pannes ou de saturation du réseau pouvant entraîner des retards, des pertes de transactions ou encore du déni de service bloquant ainsi l'accès aux \gls{actif}s des utilisateurs. 
Finalement, ces plate-formes sont soumises à la réglementation et à la surveillance des autorités financières, limitant leur accessibilité dans certains pays ou régions. 
Ce point signifie aussi que les \gls{actif}s de l'utilisateurs sont traçables par les autorités. 


\subsubsection{Fonctionnement}
Ces plate-formes fonctionnent sur le principe de l'\textit{order book method} (méthode du carnet d'ordre\cite{orderBook}), une modélisation des ordres d'achats et de vente des jetons.
Un ordre étant une demande d'un utilisateur visant à réaliser une opération à un prix et une quantité donnée. 
Cette méthode comprend deux parties: l'offre et la demande. L'offre regroupe les ordres d'achats émis par des utilisateurs sur la plate-forme et la demande, les ordres de vente.
Lors d'un dépôt, l'utilisateur s'étant au préalable enregistré sur la plate-forme, il va déposer les fonds souhaités dans un porte monnaie. 
La plate-forme va ensuite crée un \textit{IOU}\footnote{I Owe You, c'est la dette de la plate-forme envers l'utilisateur permettant de bloquer la valeur de la monnaie déposée par l'utilisateur\cite{IOU}} 
ce dernier sera échangé contre le crypto-\gls{actif} souhaité lors d'un échange ou d'une vente. \\ 
\begin{figure}[h!]
    \centering
    \includegraphics[scale=0.5]{centralisation/echange.png}
    \label{fig:simplifiedcex}
    \caption{Échange d'un jeton en Euros}
\end{figure}
Dans le cadre des échanges inter-\gls{blockchain}s, les plate-formes d'échanges utilisent des \textit{bridges} reliant les différentes \textit{\gls{blockchain}s}. 
Ces protocoles seront explicités dans la partie suivante.
Cependant, nous n'avons pas pu trouver de plus amples explications quant aux fonctionnements des plate-formes, notamment les protocoles précis utilisés lors des échanges. 
Les documentations disponibles pour les plateformes d'échanges étant à destination des utilisateurs finaux. 




\subsection{Les Blockchain Bridges}
%Auteure : Eloïse Rotondo

\subsubsection{Fonctionnement}

Comme son nom l’indique, un \textit{\gls{blockchain} bridge} également appelé \textit{\gls{cross-chain} bridge} est un protocole reliant deux \textit{\gls{blockchain}s} entre elles de manière unilatérale ou bilatérale dans une optique d’interopérabilité.\\

Dans la but de comprendre la popularité des \textit{bridges} en tant que protocole d’interopéralité, il faut en premier lieu s’intéresser au marché de la cryptomonnaie. Actuellement \gls{Bitcoin} domine en représentant 40,5\% de ce dernier, suivie ensuite par \gls{Ethereum} avec 19,5\%. Cela laisse donc 40\% du marché formé de nombreuses crytomonnaies plus petites et plus indépendantes. C’est donc naturellement, qu’une forte demande de possibilité d’échanges entre les \textit{\gls{blockchain}s} ait vu le jour de la part des utilisateurs ayant plusieurs cryptomonnaies\cite{Ngrave}.\\

Il existe trois différentes manières de déplacer les \gls{actif}s en tant que \textit{bridge}. Tout d’abord, le mécanisme de \textit{Lock and Mint} signifiant Verrouiller et Frapper, les \gls{actif}s se trouvant sur la chaîne de départ sont verrouillés sur celle-ci pour être ensuite créés sur la chaîne destinataire. Un autre mode d'échange est celui du \textit{Burnt and Mint}, ce dernier est très similaire à celui déjà présenté, la seule différence étant que les \gls{actif}s sont directement effacés plutôt que verrouillés. Pour finir, les échanges atomiques entre chaînes (Atomic Swaps) permettent un échange direct en pair-à-pair d'\gls{actif}s entre la chaîne d’origine et la chaîne de destinataire.\cite{EthereumMechanism}

\subsubsection{Mécanisme de vérification}

Comme évoqué précédemment, deux \textit{\gls{blockchain}s} ne peuvent pas communiquer directement entre elles, par conséquent lors de l’utilisation d’un \textit{bridge} les deux chaînes ne se connaissent pas et ont seulement connaissance des évènements se produisant sur leur chaîne respectives. Il est donc nécessaire d’établir une relation de confiance entre les deux chaînes pour qu’elles puissent accepter de communiquer. Pour cela, les \textit{bridges} emploient un mécanisme utilisant des \gls{vérificateur}s. Un \gls{vérificateur} est une entité connectée en tant que \gls{noeud} au réseau de la \textit{blockchain}. Ce dernier agit comme autorité de confiance, vérifiant et validant les transactions sur cette dernière. Un noeud d'une \textit{blockchain} est un ordinateur connecté au réseau de cette dernière. Un \textit{client} est un logiciel permettant de transformer un ordinateur en noeud. \cite{EthereumNodeClient} \\


Il existe un grand nombre de \textit{bridges}, chacun avec leurs propres spécificités mais ils peuvent généralement être séparés en deux catégories les \textit{Trusted \gls{blockchain} Bridge} et les \textit{Trustless \gls{blockchain} Bridge}. \\


Les \textit{Trusted Bridges} sont vérifiés de manière externe car ils utilisent un ensemble de \gls{vérificateur}s tiers pour transmettre des données entre les chaînes. Ils ont pour avantage leur rapidité, leur moindre coût et la facilité d'échange avec tous les types de données acceptés par les \textit{\gls{blockchain}s}. Cependant les \gls{vérificateur}s externes sont moins fiables que ceux de la chaîne.\cite{EthereumBridges}

Les \textit{Trustless Bridges} sont désignés comme \textit{trustless} car ils dépendent des chaînes dont ils font l’intermédiaire pour transférer des données ou des \gls{actif}s. Par conséquent leur niveau de fiabilité est égal à celui des \textit{\gls{blockchain}s} et il n’est pas nécessaire de faire confiance à un ensemble de \gls{vérificateur}s tiers (contrairement aux autres \textit{bridges}). Pour cette raison, ils sont reconnus comme étant plus fiables que les \textit{trusted bridges}.\cite{EthereumBridges}\\


 Les \textit{bridges} peuvent également être distingués en fonction de leur type de vérification. Les plus connus sont la vérification native, externe et locale.\cite{InteroperabilityBhuptani} \\

La vérification native commence par l’utilisation d’un \gls{noeud léger}. \cite{NomadDocsNative} Un noeud léger aussi connu sous le nom de client léger est un logiciel permettant de connecter les noeuds des \textit{\gls{blockchain}s} entre elles. Il est codé comme un \textit{\gls{smart contract}} puis est employé de la chaîne de l'expéditeur vers la machine virtuelle de la chaîne destinataire. Si les \gls{vérificateur}s de données de la chaîne de l'expéditeur agissent de manière correcte alors le noeud léger est vu comme véridique par la chaîne réceptionnant les \gls{actif}s ou données et peut être utilisé de manière bilatérale.
 Un avantage de cette solution est qu’elle est reconnue comme étant celle reposant le moins sur la confiance parmi celles existantes car les chaînes ne se fient qu’à leurs propres \gls{vérificateur}s pour effectuer le \textit{bridge}. Un autre bénéfice de ce mécanisme est le fait qu’il n’utilise pas de \gls{vérificateur}s tiers entre les deux \textit{\gls{blockchain}s} et donc la sécuté du réseau dépend des \textit{\gls{blockchain}s} elles-même (ce qui est avantageux car elles sont robutes et préparées aux attaques comme la chaîne d’Ethereum par exemple).
 Un désavantage de cette méthode est que le noeud léger doit être adapté aux consensus des chaînes auquelles il est attaché ce qui le rend inutilisable avec des chaînes différentes. Le noeud léger nécessite également de la maintenance en cas de changement des règles consensus (utilisées pour valider les transactions). Un autre inconvénient découlant du fait que le noeud léger est programmé de manière spécifique est que ce dernier n’est donc pas réutilisable. \\

\pagebreak

La vérification externe consiste en un ensemble de \gls{vérificateur}s n’appartenant pas aux \textit{\gls{blockchain}s} relayant les données entre les deux extrémités du \textit{bridge}. Pour se faire, un certains nombre de \gls{vérificateur}s doivent signer un message provenant de la chaîne d’envoi pour que le chaîne destinataire le reconnaisse comme valide. Par exemple, pour le \textit{bridge} \gls{Wormhole} 13 \gls{vérificateur}s sur 19 doivent avoir signé\cite{NomadDocsExternal}. Ce concept est une primitive cryptographique (algorithme cryptographique de bas niveau servant de base à un système de sécurité informatique) nommée le système de signature à seuil (désignée par TSS pour \textit{Threshold Signature Scheme})\cite{BinanceTSS}. 
Contrairement à la vérifications native, les \textit{bridges} vérifiés de manière externe sont faciles à développer, peuvent être réutilisés sans problèmes et leur maintenance coûte peu. Le désavantage conséquent de cette méthode est que la sécurité dépend des \gls{vérificateur}s tiers du pont ce qui peut fragiliser le système car ils sont généralement moins sécurisés que ceux des \textit{\gls{blockchain}s}. \\


\begin{figure}[h!]
    \centering
\stackunder{
\includegraphics[scale=0.60]{centralisation/imagesBridges/LightClient.png}}
    {\scriptsize
            Source: \url{https://docs.nomad.xyz/the-nomad-protocol/verification-mechanisms/native-verification}}
    \caption{Mécanisme utilisant un noeud léger.}
    \label{fig:LightClient}
\end{figure}

\pagebreak

La vérification externe consiste en un ensemble de \gls{vérificateur}s n’appartenant pas aux \textit{\gls{blockchain}s} relayant les données entre les deux extrémités du \textit{bridge}. Pour se faire, un certains nombre de \gls{vérificateur}s doivent signer un message provenant de la chaîne d’envoi pour que le chaîne destinataire le reconnaisse comme valide. Par exemple, pour le \textit{bridge} Wormhole 13 \gls{vérificateur}s sur 19 doivent avoir signé\cite{NomadDocsExternal}. Ce concept est une primitive cryptographique (algorithme cryptographique de bas niveau servant de base à un système de sécurité informatique) nommée le système de signature à seuil (désignée par TSS pour \textit{Threshold Signature Scheme})\cite{BinanceTSS}. 
Contrairement à la vérifications native, les \textit{bridges} vérifiés de manière externe sont faciles à développer, peuvent être réutilisés sans problèmes et leur maintenance coûte peu. Le désavantage conséquent de cette méthode est que le bon fonctionnement du système dépend des \gls{vérificateur}s tiers ce qui peut le fragiliser car ils sont généralement moins fiables que ceux des \textit{\gls{blockchain}s}. \\

Il est intéressant de noter que les \textit{\gls{blockchain}s} ont également leur propre ensemble de \gls{vérificateur}s sous la forme de \gls{vérificateur} de données. Ces derniers sont utilisés lors de la vérification locale. Lors de la vérification locale, les chaînes se vérifient entre elles en utilisant un \gls{vérificateur} en tant que représentant.

\begin{figure}[h]
    \centering
\includegraphics[scale=0.60]{centralisation/imagesBridges/DiagrammeResumeVerif.png}
\caption{Résumé des types de vérification. (locale, native et externe)}
\label{fig:LocaleVerif}
\end{figure}

\subsubsection{Les risques liés aux Bridges}

La popularité des \textit{\gls{blockchain} Bridges} pour les échanges centralisés ne cesse d’augmenter au fils du temps mais il est important de comprendre que comme tout outil, ces derniers ne sont pas sans risques. \\

Les \textit{bridges trustless} utilisent des \textit{\gls{smart contract}s} lors du processus d’échange dans le but de le rendre autonome afin de ne pas utiliser une entité centrale entre les deux \gls{blockchain}s. Cependant ces \textit{bridges} sont néanmoins centralisés car ils utilisent les \gls{vérificateur}s pour obtenir un consensus lors des transactions.
Un \textit{\gls{smart contract}} étant un script écrit par un développeur, il est possible que certaines erreurs puissent s’être glissées dans le code par inadvertance ou bien qu’il existe des failles dans le programme permettant aux attaquants de le détourner pour un profit personnel. 
Pour minimiser ce type de risques, il est recommandé d’effectuer des audits sur les \textit{bridges.} \\

Une faiblesse spécifique des \textit{bridges trusted} repose sur le fait que les utilisateurs doivent léguer le contrôle de leurs \gls{actif}s et faire confiance aux \gls{vérificateur}s externes aux \gls{blockchain}s. Sauf que dans certains cas, ces derniers peuvent coopérer pour tromper les utilisateurs en récupérant leurs \gls{actif}s puis en disparaissant comme dans les \textit{rug pull}\cite{EthereumRisks}. Ce modèle d’escroquerie peut être scindé en deux catégorie : les \textit{hard rug pull} et les \textit{soft rug pull}\cite{Hacken}. Le premier cas est basé sur un piège présent dans le code d’un \textit{\gls{smart contract}} empêchant les utilisateurs d’utiliser ou revendre les \gls{actif}s frappés, seul le fraudeur en a le droit. Il peut donc en toute tranquillité revendre les \gls{actif}s et récupérer l’argent. En revanche, pour les \textit{soft rug pull}, les utilisateurs ne sont pas coincés avec des \gls{actif}s verrouillés mais les fraudeurs utilisent des techniques psychologiques. En effet, les escrocs rendent attirant leur projet pour que les clients investissent et hésitent à se retirer par peur de perdre leur investissent (souvent de taille conséquent) puis les créateurs de la fraude disparaissent avec leurs \gls{actif}s.\\

Comme vu dans la section présentant les différentes méthodes d’échange des \textit{bridges}, ces derniers frappent les \gls{actif}s désirés sur la chaîne destinataire. Certains attaquants peuvent profiter de ce mécanisme de frappe pour effectuer ce qu’on appelle une \textit{Infinite Mint Attack}.\cite{ChainLinkRisks} Cette attaque peut se résumer à un \textit{hacker} générant un nombre élevé d’\gls{actif}s en utilisant une faille d’un \textit{bridge} sans verrouiller ou brûler d’\gls{actif}s sur sa \textit{\gls{blockchain}}. Suite à cela, l’individu réintroduit ces \gls{actif}s sur le marché ce qui fait violemment baisser leur coût ce qui engendre un risque financier systémique.\\

Les \textit{\gls{blockchain} Bridges} sont devenus un outil indispensable des échanges centralisés très rapidement, mais il ne faut pas oublier que ces protocoles sont relativement récents. Créés par de petites \gls{blockchain}s comme Syscoin et NEAR Protocol dans le but de rentre leurs chaînes interopérables avec les applications décentralisées d’\gls{Ethereum}, les premiers bridges datent de 2020\cite{Bitstamp}. Par conséquent, nous ne connaissons pas encore le comportement des \textit{bridges} lorsqu’ils font face à des scénarios sortants de la norme comme des attaques réseaux, un retour en arrière sur les transactions d’une \gls{blockchain} (souvent désigné par le terme \textit{rollback}) ou bien pendant une congestion du réseau. Ces zones d’incertitudes peuvent donc être une source de risques. \\

\begin{figure}[h!]
    \centering
\includegraphics[scale=0.30]{centralisation/imagesBridges/GraphLossesBridges.png}
    {\scriptsize
            Source: \url{https://www.treehouse.finance/insights/blockchain-and-interoperability-globalization-3-0}}
    \caption{Pertes en millions de dollars des bridges les plus connus.}
    \label{fig:GraphBridges}
\end{figure}

\subsection{Le trilemme de l’interopérabilité}

Malgré l’existence de plus d’une centaine de \textit{bridges} différents, les développeurs et les utilisateurs voulant utiliser un \textit{bridge} doivent faire des concessions lors de leur choix vis-à-vis des trois notions de \textit{trustless}, d’extensibilité (\textit{extensible}) et de généralisation (\textit{generalizable}).\\

Le mot trustless peut être traduit par «sans confiance». Si un \textit{bridge} est caractérisé comme \textit{trustless}, cela signifie que celui-ci possède un niveau équivalent à celui d’une ou des chaînes sous-jacentes, il est donc pas nécessaire de faire confiance à une entité externe aux \textit{blockchains}.  La notion  d’extensibilité signifie que le \textit{bridge} est compatible un grand nombre de chaînes.
Un \textit{bridge} respecte la notion de généralisation s’il est capable d'échanger n’importe quel type de données accepté par les deux chaînes.\\

Pour illustrer ces termes, il est possible de les appliquer aux types de vérification appartenant aux \textit{bridges}. La vérification locale respecte les notions d’extensibilité et de \textit{trustless} puisque qu'elle est applicable sur tous les \textit{bridges} peu importe les chaînes reliées et le niveau de fiabilité dépend de la chaîne la plus faible.
La vérification native ne respecte pas la notion d’extensibilité car le \textit{bridge} n’est pas réutilisable. Néanmoins elle respecte la notion de généralisation parce qu’elle est codée de manière spécifique aux \textit{\gls{blockchain}s} reliées au \textit{bridge}. Le critère basé sur la notion \textit{trustless} est également rempli étant donné que le niveau de fiabilité dépend des \gls{vérificateur}s des chaînes.
La vérification externe ne respecte pas la notion de \textit{trustless} cependant elle est fortement extensible et générale vis-à-vis des données. \cite{Ngrave}

 Suite au paragraphe précédent, il est possible de constater que les \textit{bridges} interopérables ne respectent que deux des trois notions énoncées. Ce problème est connu sous le nom de trilemme de l’interopérabilité. 

\subsubsection{Une solution optimiste}

Une solution proposée pour résoudre ce trilemme est un \textit{optimistic bridge}(\textit{bridge} optimiste) nommé vis-à-vis de sa vérification portant le même nom\cite{OptimisticBhuptani}. En effet, contrairement aux \textit{bridges} vérifiés de manière native, locale ou externe, la vérification optimiste dépend de l’utilisation d’une latence lors de la confirmation du transfert des \gls{actif}s entre les \textit{\gls{blockchain}s}. Cela priorise donc la sécurité au détriment de la vivacité, puisque les transactions sont par conséquent plus lentes mais sécurisées car le \textit{bridge} est \textit{trustless}. \\

Voici plus en détails le déroulement du processus de vérification optimiste d’un \textit{bridge}.
Ceci commence par l’envoi d'une demande de transaction de la part d’un utilisateur ou d’une application décentralisée vers la \textit{\gls{blockchain}} native par le biais d’une fonction contrat. Cette demande est ensuite acceptée par un validateur (une entité équivalente à un \gls{vérificateur}, la seule différence étant leur rôle) puis inscrite sur un \textit{block} de la chaîne.\cite{NomadDocsVerification}
Ensuite un \gls{vérificateur} a pour rôle de vérifier la transaction. Pour cela, le \gls{vérificateur} signe le haché des données envoyées précédemment. 
Suite à cela, n’importe quel système de relais peut lire le haché signé sur la chaîne originelle et l’inscrire sur une ou plusieurs chaînes destinataire. Les validateurs de la chaîne destinataire valide et l'inscrivent sur leur chaîne. Cette action déclenche alors une latence de trente minutes pendant laquelle un observateur peut signaler et prouver une fraude effectuée par la chaîne native ce qui déconnectera la communication avec la chaîne destinataire. 

Deux scénarios sont alors possibles. Premier cas, si aucun observateur ne se manifeste, les données de la transaction sont finalisées puis traitées par la chaîne destinataire. Les validateurs sont récompensés pour leur travail avec une partie des frais de transaction. \cite{Fees} Deuxième cas, un observateur prouve une fraude pendant les trente minutes accordées. Le \gls{vérificateur} ayant fraudé est pénalisé par la perte de sa récompense (qui sera obtenu par l'observateur) et son exclusion du réseau.\cite{EthereumSlashing}

\subsubsection{Possibles faiblesses de l’optimisme et leurs solutions}

Le bon fonctionnement des \textit{bridges} optimistes dépend des chaînes auquelles il est rattaché donc tant que ces dernières sont protégées correctement la seule conséquence d’une faille du \textit{bridge} est l’arrêt système plutôt qu’une perte de fonds comme cela peut être le cas avec les autres types de \textit{bridge}.

Les deux acteurs principaux ayant les moyens de nuire au bon fonctionnement du \textit{bridge} ainsi que de la transaction est l’agent \gls{vérificateur} ainsi que l’observateur car ces deux rôles ont de l’influence sur cette dernière. \\

Le premier cas impliquant le \gls{vérificateur} se nommant \textit{Updater Fraud} (fraude du \gls{vérificateur})  fut déjà mentionné lors de la présentation du fonctionnement du \textit{bridge} optimiste. Ce dernier repose sur le fait que toute transaction doit passer par le \gls{vérificateur} et que par conséquent toute fraude est originaire de ce dernier. Sinon si cela venait d’un autre participant, le \gls{vérificateur} n’aurait alors pas accepté la transaction. C’est pourquoi, lors de l’intervention d’un observateur prouvant une fraude, le \gls{vérificateur} est sanctionné par le retrait sur son solde d'un montant équivalent à la récompense promise et par son exclusion du réseau de la \textit{blockchain}. \\

La seconde faiblesse liée aux \gls{vérificateur}s est un \textit{Updater DoS} ou déni de services de la part du \gls{vérificateur}. En effet, il est possible que le processus soit interrompu si un validateur arrête de signer empêchant l'échange inter-chaînes de se produire.
Une solution a été implémentée pour palier à cela comme la mise en place d’un système de substitution avec la présence de plusieurs \gls{vérificateur}s sur une même chaîne afin de pouvoir prendre le relai en cas de manque de réponse de la part de celui étant rattaché au transfert.  Pour éviter que ce scénario se produise fréquemment le \gls{vérificateur} ayant manqué son tour lors de la signature (que cela soit accidentel ou voulu) est pénalisé de la même manière que le cas précédent. \\

Maintenant que les possibles obstacles au bon fonctionnement du \textit{bridge} liés aux \gls{vérificateur}s ont été mis en lumière, il est également possible que l’observateur ait un comportement malveillant.  Effectivement, malgré l’absence de tromperie (puisque le \gls{vérificateur} remplit son rôle), l’observateur peut abuser du mécanisme de déclaration de fraude pour impacter le bon déroulement du procédé. \\

La faculté de l’observateur à pouvoir couper la connexion s’il conteste la transaction lui permet d’effectuer un déni de service appelé \textit{Watcher DoS}. C’est pourquoi il lui ait possible de fermer définitivement la connexion d’une transaction si ce dernier continue sans cesse de couper le processus sans raison valable. Heureusement, la fermeture ne concerne que la connexion et n’impacte en aucun cas le système du \textit{bridge}. Cependant cette attaque semble irrationnelle en terme de ressources et de temps car l’observateur effectuant le déni de service ne gagne rien financièrement contrairement au processus habituel. En effet, si un observateur prouve une fraude correctement, ce dernier peut récupérer la récompense du \gls{vérificateur}. Mais ici puisqu’aucune fraude n’est prouvée les données se trouvant sur la chaîne d’origine sont conservées et sécurisés. Cela cause seulement une perte de temps pour l’utilisateur ou l’application décentralisée voulant effectuer l'échange d’une \textit{\gls{blockchain}} à une autre.

Une réponse à ce problème actuellement mise en œuvre par le \textit{bridge} de \gls{Nomad} est la présence d’un groupe restreint d’observateurs autorisés à contester, de cette manière il est facile de connaître les observateurs malveillants. Chaque observateur possède une clé permettant de signer une attestation confirmant la présence d’une fraude dans la transaction, chaque \textit{bridge} stocke un ensemble contenant les adresses des attestations appartenant aux observateurs autorisés. Si l’attestation reçue par le \textit{bridge} est présente dans l’ensemble alors la connexion est rompue\cite{NomadDocsWatcher}.
Sur le long terme, une proposition consistant à la mise en place de frais si l’on souhaite contester est en train d’être étudiée. Le montant doit répondre à deux contraintes: ce dernier doit être assez haut pour dissuader les observateurs malhonnêtes mais assez bas pour que ceux ayant réellement l’envie de prouver de manière valide une fraude existante puissent le faire. Dans la continuité de cette solution, il serait également possible de récupérer la signature de la déconnexion générée par l’observateur sur la chaîne originale et de le pénaliser en lui retirant les frais qu’il a payé tel une garantie\cite{OptimisticBhuptani}.



\subsection{Wormhole}
En 2017, une cryptomonnaie adossée à la \textit{blockchain} Solana a émergée avec des caractéristiques
similaires à Ethereum : \textit{blockchain} publique, \textit{smart contracts}.\\
Solana est devenue de facto une \textit{blockchain} concurrente à Ethereum et est aujourd'hui 
la onzième \textit{blockchain} en terme de capitalisation selon l'aggrégateur de marché Coinmarketcap.\\
Un besoin d'échanger des actifs entre les \textit{blockchains} Ethereum et Solana est apparu, 
d'où l'introduction en 2020 de la première version de Wormhole.
Initialement, Wormhole v1 a été concu comme un \textit{bridge} entre Ethereum et Solana.
Depuis, Wormhole s'est développé au-delà de Solana avec le lancement d'une deuxième version en 2021 
en tant que protocole générique de passage de messages.\\
À l'écriture de ce rapport, 22 \cite{wormholeNetwork} \textit{blockchains} sont compatibles avec Wormhole 
dont : BNBChain, Ethereum, Moonbeam, Polygon, Solana...\\
Le protocole émet un message à partir d'une \textit{blockchain} source qui est validé par un réseau de 
gardiens.\\ 
Le message est ensuite envoyé à la \textit{blockchain} cible pour être traité.

\subsubsection{VAA (\textit{Verified action approval})}

Lorsqu'un \textit{smart contract} envoie un message \textit{crosschain} comme un verrouillage
de jetons sur une \textit{blockchain} source et une demande de frappe de jetons sur une 
\textit{blockchain} cible, celui-ci interargit avec un \textit{core contract} \cite{wormholeCoreContract}.
Un \textit{core contract} est déployé sur toutes les \textit{blockchains} compatibles avec le protocole 
Wormhole. Tout \textit{core contract} est observé par le réseau de gardiens.
Un message Wormhole est émis grâce à la fonction \textit{publishMessage()} prenant en entrée le \textit{payload}.
La sortie de cette fonction est un \textit{sequence number}, un numéro d'index unique pour le message.
Combiné à l'adresse du contrat de l'émetteur et à l'identifiant de la chaîne de l'émetteur, le message 
correspondant peut être récupéré auprès d'un nœud du réseau de gardiens.\\
Un message Wormhole est vérifié grâce à la fonction \textit{parseAndVerifyVAA()} prenant en entrée le message.
Selon la validité de l'entrée, la fonction retourne en sortie le \textit{payload} ou une exception.
\newpage

VAA \cite{wormholeVAA} est la primitive de messagerie de base de Wormhole. Un VAA contient une en-tête 
ainsi qu'un \textit{body}. L'en-tête contient l'index des gardiens ayant signés le message et la collection des signatures.
L'en-tête permet au \textit{core contract} de vérifier l'authenticité du VAA.
Quant au \textit{body}, il contient des informations comme le numéro d'identification de la chaîne 
Wormhole du contrat émetteur, l'adresse du contrat émetteur, le \textit{sequence number} 
et le \textit{payload}.\\ 5 \textit{payloads} peuvent être utilisés dont \textit{Transfer} et 
\textit{AssetMeta}, attestant les méta-données du jeton.\\
Le \textit{payload AssetMeta} est obligatoire avant un premier transfert.
En effet, le \textit{payload Transfer} n'informe pas la chaîne B des meta-données du jeton verrouillé.
En l'absence de connaissance de ces informations, il n'est pas possible pour la \textit{blockchain} B 
de frapper la quantité correcte de jetons.\\
Si l'on souhaite ensuite transférer des jetons depuis une \textit{blockchain} A vers une 
\textit{blockchain} B, il faut verrouiller les jetons sur A et les frapper sur B.
D'où l'utilisation du \textit{payload Transfer} contenant des informations comme la 
quantité de jetons transférés, l'adresse de la chaîne d'origine et de destination, 
le numéro d'identification de la chaîne d'origine et de destination..
Une preuve doit être fournie que les jetons sur A sont verrouillés avant que la frappe puisse 
avoir lieu sur B. La signature des gardiens sur le VAA correspondant est la preuve apportée à 
la \textit{blockchain} B que le verrouillage a bien été effectué et que la frappe de jetons sur 
B est légitime.

\subsubsection{Gardiens}

Un gardien \cite{wormholeGuardian} est une autorité de confiance qui a comme de rôle valider 
(par une signature) le \textit{payload} contenu dans un VAA.
Comme évoqué précédemment, le réseau de gardiens observe tous les messages \textit{crosschain} via la 
surveillance des \textit{core contracts}.
Le réseau de gardiens est composé de 19 gardiens à parts égales sans chef (\textit{leaderless}).
Il est conçu pour servir d'oracle à Wormhole et est l'élement le plus critique de l'écosystème.
Si une majorité de deux tiers ou plus des gardiens signent le même VAA, alors le consensus est atteint : 
le VAA est automatiquement considéré valide par  tous les contrats Wormhole sur toutes les 
\textit{blockchains} et le \textit{payload} est actionné. 
Chaque gardien utilise un algorithme de signature à courbe elliptique : ECSDA pour 
\textit{Elliptic Curve Signature Digital Algorithm}.
Plus précisément, chaque gardien se réfère à «secp256k1» comme paramètres de la courbe elliptique, 
aussi utilisé par les \textit{blockchains} Bitcoin et Ethereum.\\
Le modèle de consensus utilisé est une \textit{Proof of Authority} (PoA) avec un système de 
\textit{multisignature} M/N \cite{wormholeChainswap}, c'est à dire que M clefs parmi N sont nécessaires 
pour signer un VAA. Ce modèle permet un traitement rapide des transactions et une dispense de participation monétaire, par rapport à la preuve de travail (PoW) et la preuve 
de participation (PoS). Cependant, il présente également des désavantages : le système est par 
\textit{design} centralisé et dépend d'un petit groupe de nœuds pouvant créer un point de 
défaillance unique par l'utilisation commune d'une fonction vulnérable. Il est questionnable de restaurer des tiers de confiance dans le cadre d'un système 
devenu populaire grâce à l'absence de tels autorités. Wormhole justifie la décentralisation de leur 
système \cite{wormholeGuardian} par la présence de plusieurs parties (et non d'un seul) dans le contrôle du réseau.
Selon notre analyse, la décentralisation résulte de l'absence d'un ou plusieurs tier(s) de confiance lorsque deux parties 
souhaitent réaliser une transaction.
\newpage

\subsubsection{Relais}

Un relai \cite{wormholeRelayer} est un processus qui délivre un ou plusieurs VAA(s) à une destination.
Les relais ne sont ni de confiance, ni privilégiés, ils écoutent directement le réseau de gardiens 
via l'intermédiaire d'un processus espion. Ces relais ne peuvent pas compromettre l'intégrité d'un VAA 
car une altération serait détectée lors du processus de vérification des signatures. Cependant, il n'est 
pas assuré qu'un relai transmette un VAA à destination, d'où une perte de disponibilité. Il est conseillé
d'héberger soi-même ces relais pour supporter son application.

\begin{figure}[h!]
    \centering
    \includegraphics[scale=0.5]{centralisation/uml_design_v2.png}
    \label{fig:wormholeDesign}
    \caption{Architecture Wormhole \cite{wormholeArch}}
  \end{figure}

% @startuml
% rectangle r1 as "Source Token Bridge 
% Relayer Contract"
% rectangle r2 as "Source Token 
% Bridge"
% rectangle r3 as "Source Core 
% Contract"
% storage "Guardians" as r4
% hexagon r5 as "Off-chain 
% Message Relayer" 
% rectangle r6 as "Target Token Bridge 
% Relayer Contract"
% rectangle r7 as "Target Token 
% Bridge"
% rectangle r8 as "Target  Core 
% Contract"

% r1 -> r2 : Transfer()
% r2 -> r3 : publishMessage()
% r3 -do-> r4 : Guardien reads message
% r4 -left-> r5 : Signed message

% r5 -do-> r6 : Signed message
% r6 -> r5 : Relayer fee

% r6 -> r7 : Signed message
% r7 -> r6 : Wrapped token

% r7 -> r8 : parseAndVerifyVAA()
% @enduml




\subsection{Analyse d'attaques}
% Auteur Romain TESTUD
\subsubsection{Mise en contexte}
Les \textit{\gls{blockchain}s} et leurs protocoles d'échanges ne sont pas exemptes d'attaques informatiques ou bien de défaillances.
Ces attaques peuvent cibler des portefeuilles (attaques sur des \textit{hot wallets}\footnote{portefeuille de cryptomonnaies en ligne, à différencier des \textit{Cold Wallets}, des portefeuilles hors lignes}) ou encore des \textit{bridges}. 
Ce sont ces dernières qui nous ont intéressées dans le cadre de ce projet de recherche sur les échanges inter-\gls{blockchain}s. 
Les bridges, comme explicité dans la partie dédiée du rapport, sont des protocoles permettant la circulation de données entre deux \textit{\gls{blockchain}s} différentes.\\
Nous avons, au cours de nos recherches, trouvés de nombreux cas d'attaques sur des protocoles d'échanges inter-\gls{blockchain}s. 
De manière à illustrer les types d'attaques possibles et les points critiques de ces systèmes nous allons décrire deux attaques parmi les plus importantes : \textit{Wormhole} et \textit{Nomad}.

\subsubsection{Le cas Wormhole}
Nous vous avons présenté le protocole \textit{Wormhole} dans la partie précédente. 
Le 2 Février 2022, une attaque exploite une erreur d'implémentation dans une \textit{\gls{dApp}} sur la chaîne Solana \cite{SolMed} \cite{SolRekt}. 
Pour se faire l'attaquant à réussi à contourner la vérification des signatures des gardiens en exploitant
une correction de bug ayant été publié sur le code source du projet mais n'étant pas encore effective en production.
Ainsi il à réussi à récupérer l'équivalent de 120 000 \textit{ETH} en \textit{whETH} (\textit{Wormhole ETH}). 
Lors d'un transfert de jetons d'une chaîne à une autre, plusieurs étapes sont réalisées par différentes fonctions.
Après la formulation de la transaction, une fonction va se charger de récupérer les signatures des gardiens dans un \textit{SignatureSet}\footnote{Ensemble de signatures de gardiens}, ces dernières sont ensuite vérifiées. 
Pour cela, une fonction nommée \texttt{verify\_signature} va appeler un programme de vérification de Solana permettant l'analyse du \textit{SignatureSet}. 
L'appel à ce programme se fait de la manière suivante, en utilisant le nom \texttt{sysvarinstruction} \cite{SolGitError} dans la transaction. 
Dès lors que les signatures sont validées, un \textit{VAA} peut être émis et transmis vers la \textit{\gls{blockchain}} souhaitée. \\
La transaction de l'attaquant étant frauduleuse, il n'aurait donc pas pu obtenir de signatures des gardiens. 
Pour contourner cette étape de récupération des signatures la transaction de l'attaquant était dotée d'un \textit{SignatureSet} correspondant à une transaction antérieure. 
Seulement, n'étant pas pour la bonne opération cet ensemble ne peut pas être approuvé par \texttt{verify\_signature}. 
C'est ici que l'attaquant à utilisé un défaut d'implémentation pour valider son \textit{SignatureSet}. 
Comme décrit précédemment, la fonction \texttt{verify\_signature} appelle un programme pour effectuer la vérification des signatures. 
Cependant il n'y à pas de vérification faites sur le programme appelé, l'attaquant à pu donc utiliser une adresse différente lui permettant de valider sa transaction. 
Avec le  \texttt{SignatureSet} ainsi validé, l'attaquant a pu générer un \textit{VAA} valide et pu déclencher une frappe de jeton vers son propre compte sans en avoir déposé au préalable. 
La correction de cette faille était contenue dans la mise à jour évoquée en début de paragraphe\cite{SolGitFixed}, permettant la vérification du programme appelé pour la vérification. 

\subsubsection{Le cas Nomad}
Nomad est un protocole d'interopérabilité entre chaînes permettant de passer des \gls{actif}s entre deux \textit{\gls{blockchain}s} différentes. 
Pour fonctionner, ce protocole fait appel à des applications décentralisées opérant sur les chaînes du réseau. 
Une première \textit{\gls{dApp}} appelée \textit{réplica} est déployée sur les \textit{\gls{blockchain}s} recevant les messages, elle fait office de "boite de réception". 
Une seconde \textit{\gls{dApp}} appelée \textit{home} est déployé sur les \textit{\gls{blockchain}s} émettrices de message. \\
Le 1\textsuperscript{er} août 2022 une attaque exploitant une erreur d'implémentation sur l'application \textit{Réplica} a engendré une perte de 190 millions de dollars en liquidité \cite{NomadMedium} \cite{NomadRekt}.
Cette attaque s'est déroulée après le déploiement d'une mise à jour, un moyen de contourner la vérification des signatures du message étant apparu. 
En analysant l'application \textit{Réplica} après la mise à jour, nous pouvons voir que lors d'une initialisation, la racine des messages, appelée \texttt{\_commitedRoot}, est initialisée à $0$, ce signifiant que le message n'a pas encore été validé. 
\begin{lstlisting}[caption={Fonction \textit{initialize} de \textit{Réplica} contenant une erreur \cite{NomadGitError}}]
    function initialize(
        uint32 _remoteDomain,
        address _updater,
        bytes32 _committedRoot,
        uint256 _optimisticSeconds
    ) public initializer {
        __NomadBase_initialize(_updater);
        // set storage variables
        entered = 1;
        remoteDomain = _remoteDomain;
        committedRoot = _committedRoot;
        // pre-approve the committed root.
        confirmAt[_committedRoot] = 1;
        _setOptimisticTimeout(_optimisticSeconds);
    }
\end{lstlisting}

Dans les lignes précédentes nous observons cette affectation : \texttt{confirmAt[\_commitedRoot] = 1}, le rôle de cette ligne est de pré-approuver la racine d'un message. 
Cette fonction est utilisée pour approuver le premier message lors du déploiement du contrat sur une \textit{\gls{blockchain}}. 
Or ici, la valeur de la racine à été initialisée a $0$, donc cette racine devient une racine valide pour la fonction de vérification des messages. 
Seulement comme nous l'avons énoncé précédemment, $0$ est la valeur par défaut pour un message n'ayant pas encore été vérifié. 
Ainsi, lors de l'émission d'un message par la fonction \texttt{process}, tout message non vérifié sera envoyé. 
Cette erreur d'implémentation a permis à des pirates d'effectuer plusieurs transactions frauduleuses et de retirer l'équivalent de 190 Millions de dollars dans la réserve de liquidité du bridge de Nomad. 
Le contrat à été corrigé, dans une mise en ligne datant du 3 Septembre 2022, tel que la racine $0$ n'est plus pré-approuvée. 

\begin{lstlisting}[caption={Fonction corrigée de l'application \textit{Réplica} \cite{NomadGitFixed}}]
    function initialize(
        uint32 _remoteDomain,
        address _updater,
        bytes32 _committedRoot,
        uint256 _optimisticSeconds
    ) public initializer {
        __NomadBase_initialize(_updater);
        // set storage variables
        entered = 1;
        remoteDomain = _remoteDomain;
        committedRoot = _committedRoot;
        // pre-approve the committed root.
        if (_committedRoot != bytes32(0)) confirmAt[_committedRoot] = 1;
        _setOptimisticTimeout(_optimisticSeconds);
    }
\end{lstlisting}


%\subsubsection{DeFi hacklabs}
%Lors de nos recherches sur des attaques sur des protocoles inter-\gls{blockchain}s, nous avons découvert \textit{Web3sec}, un groupe de recherche centré sur la sécurité du web3. 
%Le groupe met à disposition des ressources indexés sur une page notion (en annexe) :
%\begin{itemize}
%    \item Plusieurs dépots \textit{Github} pour étudier les attaques et apprendre les vulnérabilités sur ces types de programmes.
%    \item \textit{DeFi Hacks Analysis - Root Cause} : Une base de données d'analyses d'attaques sur des solutions et organismes traitant sur des \gls{blockchain}s, les analyses sont sourcées et redirigent vers le dépot \textit{GitHub} de reproduction des attaques.
%    \item Un ensemble d'outils utilitaires tels que des outils de debug de transaction, des dashboards ou encore des newletters.
%\end{itemize}
%Ces outils nous ont permis d'explorer de nombreuses sources concernant les attaques sur les protocoles inter\gls{blockchain}s.

\subsubsection{Contre-mesures et solutions envisageables}
Comme nous avons pu le voir, de nombreux cas d'attaques sont observables sur des protocoles d'échanges centralisés. 
Dans la plupart des cas, elles résultent de problèmes d'implémentations et autres oublis dans les codes sources des protocoles utilisés. 
Cela peut être expliqué par le fait que la \textit{\gls{blockchain}} est un domaine qui évolue très vite et chaque innovation technique peut rapporter des parts de marché importantes au premier arrivé. 
De plus, de nombreux acteurs se spécialisent dans ce domaine sans nécessairement avoir une grande culture de la cybersécurité.
Il se peut donc que des erreurs d'implémentations paraissant évidentes ne soient pas relevés lors de la mise en production. \\ 
Des moyens de limiter le plus possible l'apparition de tels événements sont néanmoins possibles. 
Tout d'abord des standards de sécurité pourraient être mis en place afin de déterminer un socle minimal à atteindre. 
Des audits et des analyses de sécurité peuvent être mis en place pendant la production ou après la publication des protocoles, en respectant un cycle de développement "classique".
%Nous pouvons aussi nous questionner quant à l'implication du tier de confiance dans l'apparition de ces failles. 
%En effet, les attaques les plus importantes ont comme point d'entrée une faille dans la structure de l'intermédiaire. 
%Cette menace constante d'attaques a mené à des recherches visant à soutirer ce tiers des échanges inter-\gls{blockchain}. Ce menant aux échanges décentralisés.


\subsection{Les limites du centralisé}
% Auteur : Romain TESTUD
\begin{frame}{Les limites du centralisé}
    \begin{itemize}
        \item Manque de transparence.
        \item Manque de maturité en matière de sécurité.
        \item Questionnement sur le tiers de confiance
    \end{itemize}
    \pause
    \begin{block}{Le tiers de confiance}
        \begin{itemize}
            \item Source d'attaques.
            \item Point critique des CEX.
        \end{itemize}
        $\rightarrow$ Volonté de se passer d'un tiers de confiance.
    \end{block}
\end{frame}
\section{Décentralisation}
\subsection{Les Plate-formes d'échanges centralisés}
\subsubsection{Définition}
Nous avons commencé nos recherches en nous intéressant en premier lieu aux moyens d'échanges les plus répandus. 
Cela nous a mené vers les plate-formes d'échanges centralisé. 
Ce sont des plate-formes, pouvant prendre la forme d'applications web, qui permettent aux utilisateurs d'acheter, de vendre ou d'échanger des \gls{actif}s numériques contre d'autres \gls{actif}s numériques ou en monnaies fiduciaires. 
Ces plate-formes peuvent opérer sur des \textit{\gls{blockchain}s} publiques ou être dédiées à une utilisation en interne. 
Elles sont dites centralisées car elles sont gérées par une entreprise ou une organisation hiérarchisée qui contrôle les transactions et les fonds des utilisateurs.
Ces plate-formes sont donc considérées comme des tiers de confiance et agissent en tant qu'intermédiaires entre les acheteurs et les vendeurs en assurant la sécurité, la liquidité et la rapidité des transactions.
C'est la solution la plus utilisée dans le secteur des \gls{actif}s numériques, elles offrent très souvent une certaine variété de services tels que le prêt et/ou le \textit{stacking} \footnote{Stacking : Action de verrouiller des jetons en vue de recevoir des récompenses \cite{defStack}}.
Elles proposent également un large éventail de cryptomonnaies disponibles.

\subsubsection{Inconvénients et risques}
Nous avons pu tout de même relever certains inconvénients et certains risques pour les utilisateurs liés à l'utilisation de ces plate-formes. 
Tout d'abords, les utilisateurs doivent confier leurs fonds et leurs données à un tier en qui ils doivent avoir confiance. 
Cela peut exposer les utilisateurs à de la fraude, du vol ou encore du piratage si les plate-formes présentent des failles de sécurité. \\
Ensuite, ces plate-formes peuvent être victimes de pannes ou de saturation du réseau pouvant entraîner des retards, des pertes de transactions ou encore du déni de service bloquant ainsi l'accès aux \gls{actif}s des utilisateurs. 
Finalement, ces plate-formes sont soumises à la réglementation et à la surveillance des autorités financières, limitant leur accessibilité dans certains pays ou régions. 
Ce point signifie aussi que les \gls{actif}s de l'utilisateurs sont traçables par les autorités. 


\subsubsection{Fonctionnement}
Ces plate-formes fonctionnent sur le principe de l'\textit{order book method} (méthode du carnet d'ordre\cite{orderBook}), une modélisation des ordres d'achats et de vente des jetons.
Un ordre étant une demande d'un utilisateur visant à réaliser une opération à un prix et une quantité donnée. 
Cette méthode comprend deux parties: l'offre et la demande. L'offre regroupe les ordres d'achats émis par des utilisateurs sur la plate-forme et la demande, les ordres de vente.
Lors d'un dépôt, l'utilisateur s'étant au préalable enregistré sur la plate-forme, il va déposer les fonds souhaités dans un porte monnaie. 
La plate-forme va ensuite crée un \textit{IOU}\footnote{I Owe You, c'est la dette de la plate-forme envers l'utilisateur permettant de bloquer la valeur de la monnaie déposée par l'utilisateur\cite{IOU}} 
ce dernier sera échangé contre le crypto-\gls{actif} souhaité lors d'un échange ou d'une vente. \\ 
\begin{figure}[h!]
    \centering
    \includegraphics[scale=0.5]{centralisation/echange.png}
    \label{fig:simplifiedcex}
    \caption{Échange d'un jeton en Euros}
\end{figure}
Dans le cadre des échanges inter-\gls{blockchain}s, les plate-formes d'échanges utilisent des \textit{bridges} reliant les différentes \textit{\gls{blockchain}s}. 
Ces protocoles seront explicités dans la partie suivante.
Cependant, nous n'avons pas pu trouver de plus amples explications quant aux fonctionnements des plate-formes, notamment les protocoles précis utilisés lors des échanges. 
Les documentations disponibles pour les plateformes d'échanges étant à destination des utilisateurs finaux. 




\subsection{Les Blockchain Bridges}
%Auteure : Eloïse Rotondo

\subsubsection{Fonctionnement}

Comme son nom l’indique, un \textit{\gls{blockchain} bridge} également appelé \textit{\gls{cross-chain} bridge} est un protocole reliant deux \textit{\gls{blockchain}s} entre elles de manière unilatérale ou bilatérale dans une optique d’interopérabilité.\\

Dans la but de comprendre la popularité des \textit{bridges} en tant que protocole d’interopéralité, il faut en premier lieu s’intéresser au marché de la cryptomonnaie. Actuellement \gls{Bitcoin} domine en représentant 40,5\% de ce dernier, suivie ensuite par \gls{Ethereum} avec 19,5\%. Cela laisse donc 40\% du marché formé de nombreuses crytomonnaies plus petites et plus indépendantes. C’est donc naturellement, qu’une forte demande de possibilité d’échanges entre les \textit{\gls{blockchain}s} ait vu le jour de la part des utilisateurs ayant plusieurs cryptomonnaies\cite{Ngrave}.\\

Il existe trois différentes manières de déplacer les \gls{actif}s en tant que \textit{bridge}. Tout d’abord, le mécanisme de \textit{Lock and Mint} signifiant Verrouiller et Frapper, les \gls{actif}s se trouvant sur la chaîne de départ sont verrouillés sur celle-ci pour être ensuite créés sur la chaîne destinataire. Un autre mode d'échange est celui du \textit{Burnt and Mint}, ce dernier est très similaire à celui déjà présenté, la seule différence étant que les \gls{actif}s sont directement effacés plutôt que verrouillés. Pour finir, les échanges atomiques entre chaînes (Atomic Swaps) permettent un échange direct en pair-à-pair d'\gls{actif}s entre la chaîne d’origine et la chaîne de destinataire.\cite{EthereumMechanism}

\subsubsection{Mécanisme de vérification}

Comme évoqué précédemment, deux \textit{\gls{blockchain}s} ne peuvent pas communiquer directement entre elles, par conséquent lors de l’utilisation d’un \textit{bridge} les deux chaînes ne se connaissent pas et ont seulement connaissance des évènements se produisant sur leur chaîne respectives. Il est donc nécessaire d’établir une relation de confiance entre les deux chaînes pour qu’elles puissent accepter de communiquer. Pour cela, les \textit{bridges} emploient un mécanisme utilisant des \gls{vérificateur}s. Un \gls{vérificateur} est une entité connectée en tant que \gls{noeud} au réseau de la \textit{blockchain}. Ce dernier agit comme autorité de confiance, vérifiant et validant les transactions sur cette dernière. Un noeud d'une \textit{blockchain} est un ordinateur connecté au réseau de cette dernière. Un \textit{client} est un logiciel permettant de transformer un ordinateur en noeud. \cite{EthereumNodeClient} \\


Il existe un grand nombre de \textit{bridges}, chacun avec leurs propres spécificités mais ils peuvent généralement être séparés en deux catégories les \textit{Trusted \gls{blockchain} Bridge} et les \textit{Trustless \gls{blockchain} Bridge}. \\


Les \textit{Trusted Bridges} sont vérifiés de manière externe car ils utilisent un ensemble de \gls{vérificateur}s tiers pour transmettre des données entre les chaînes. Ils ont pour avantage leur rapidité, leur moindre coût et la facilité d'échange avec tous les types de données acceptés par les \textit{\gls{blockchain}s}. Cependant les \gls{vérificateur}s externes sont moins fiables que ceux de la chaîne.\cite{EthereumBridges}

Les \textit{Trustless Bridges} sont désignés comme \textit{trustless} car ils dépendent des chaînes dont ils font l’intermédiaire pour transférer des données ou des \gls{actif}s. Par conséquent leur niveau de fiabilité est égal à celui des \textit{\gls{blockchain}s} et il n’est pas nécessaire de faire confiance à un ensemble de \gls{vérificateur}s tiers (contrairement aux autres \textit{bridges}). Pour cette raison, ils sont reconnus comme étant plus fiables que les \textit{trusted bridges}.\cite{EthereumBridges}\\


 Les \textit{bridges} peuvent également être distingués en fonction de leur type de vérification. Les plus connus sont la vérification native, externe et locale.\cite{InteroperabilityBhuptani} \\

La vérification native commence par l’utilisation d’un \gls{noeud léger}. \cite{NomadDocsNative} Un noeud léger aussi connu sous le nom de client léger est un logiciel permettant de connecter les noeuds des \textit{\gls{blockchain}s} entre elles. Il est codé comme un \textit{\gls{smart contract}} puis est employé de la chaîne de l'expéditeur vers la machine virtuelle de la chaîne destinataire. Si les \gls{vérificateur}s de données de la chaîne de l'expéditeur agissent de manière correcte alors le noeud léger est vu comme véridique par la chaîne réceptionnant les \gls{actif}s ou données et peut être utilisé de manière bilatérale.
 Un avantage de cette solution est qu’elle est reconnue comme étant celle reposant le moins sur la confiance parmi celles existantes car les chaînes ne se fient qu’à leurs propres \gls{vérificateur}s pour effectuer le \textit{bridge}. Un autre bénéfice de ce mécanisme est le fait qu’il n’utilise pas de \gls{vérificateur}s tiers entre les deux \textit{\gls{blockchain}s} et donc la sécuté du réseau dépend des \textit{\gls{blockchain}s} elles-même (ce qui est avantageux car elles sont robutes et préparées aux attaques comme la chaîne d’Ethereum par exemple).
 Un désavantage de cette méthode est que le noeud léger doit être adapté aux consensus des chaînes auquelles il est attaché ce qui le rend inutilisable avec des chaînes différentes. Le noeud léger nécessite également de la maintenance en cas de changement des règles consensus (utilisées pour valider les transactions). Un autre inconvénient découlant du fait que le noeud léger est programmé de manière spécifique est que ce dernier n’est donc pas réutilisable. \\

\pagebreak

La vérification externe consiste en un ensemble de \gls{vérificateur}s n’appartenant pas aux \textit{\gls{blockchain}s} relayant les données entre les deux extrémités du \textit{bridge}. Pour se faire, un certains nombre de \gls{vérificateur}s doivent signer un message provenant de la chaîne d’envoi pour que le chaîne destinataire le reconnaisse comme valide. Par exemple, pour le \textit{bridge} \gls{Wormhole} 13 \gls{vérificateur}s sur 19 doivent avoir signé\cite{NomadDocsExternal}. Ce concept est une primitive cryptographique (algorithme cryptographique de bas niveau servant de base à un système de sécurité informatique) nommée le système de signature à seuil (désignée par TSS pour \textit{Threshold Signature Scheme})\cite{BinanceTSS}. 
Contrairement à la vérifications native, les \textit{bridges} vérifiés de manière externe sont faciles à développer, peuvent être réutilisés sans problèmes et leur maintenance coûte peu. Le désavantage conséquent de cette méthode est que la sécurité dépend des \gls{vérificateur}s tiers du pont ce qui peut fragiliser le système car ils sont généralement moins sécurisés que ceux des \textit{\gls{blockchain}s}. \\


\begin{figure}[h!]
    \centering
\stackunder{
\includegraphics[scale=0.60]{centralisation/imagesBridges/LightClient.png}}
    {\scriptsize
            Source: \url{https://docs.nomad.xyz/the-nomad-protocol/verification-mechanisms/native-verification}}
    \caption{Mécanisme utilisant un noeud léger.}
    \label{fig:LightClient}
\end{figure}

\pagebreak

La vérification externe consiste en un ensemble de \gls{vérificateur}s n’appartenant pas aux \textit{\gls{blockchain}s} relayant les données entre les deux extrémités du \textit{bridge}. Pour se faire, un certains nombre de \gls{vérificateur}s doivent signer un message provenant de la chaîne d’envoi pour que le chaîne destinataire le reconnaisse comme valide. Par exemple, pour le \textit{bridge} Wormhole 13 \gls{vérificateur}s sur 19 doivent avoir signé\cite{NomadDocsExternal}. Ce concept est une primitive cryptographique (algorithme cryptographique de bas niveau servant de base à un système de sécurité informatique) nommée le système de signature à seuil (désignée par TSS pour \textit{Threshold Signature Scheme})\cite{BinanceTSS}. 
Contrairement à la vérifications native, les \textit{bridges} vérifiés de manière externe sont faciles à développer, peuvent être réutilisés sans problèmes et leur maintenance coûte peu. Le désavantage conséquent de cette méthode est que le bon fonctionnement du système dépend des \gls{vérificateur}s tiers ce qui peut le fragiliser car ils sont généralement moins fiables que ceux des \textit{\gls{blockchain}s}. \\

Il est intéressant de noter que les \textit{\gls{blockchain}s} ont également leur propre ensemble de \gls{vérificateur}s sous la forme de \gls{vérificateur} de données. Ces derniers sont utilisés lors de la vérification locale. Lors de la vérification locale, les chaînes se vérifient entre elles en utilisant un \gls{vérificateur} en tant que représentant.

\begin{figure}[h]
    \centering
\includegraphics[scale=0.60]{centralisation/imagesBridges/DiagrammeResumeVerif.png}
\caption{Résumé des types de vérification. (locale, native et externe)}
\label{fig:LocaleVerif}
\end{figure}

\subsubsection{Les risques liés aux Bridges}

La popularité des \textit{\gls{blockchain} Bridges} pour les échanges centralisés ne cesse d’augmenter au fils du temps mais il est important de comprendre que comme tout outil, ces derniers ne sont pas sans risques. \\

Les \textit{bridges trustless} utilisent des \textit{\gls{smart contract}s} lors du processus d’échange dans le but de le rendre autonome afin de ne pas utiliser une entité centrale entre les deux \gls{blockchain}s. Cependant ces \textit{bridges} sont néanmoins centralisés car ils utilisent les \gls{vérificateur}s pour obtenir un consensus lors des transactions.
Un \textit{\gls{smart contract}} étant un script écrit par un développeur, il est possible que certaines erreurs puissent s’être glissées dans le code par inadvertance ou bien qu’il existe des failles dans le programme permettant aux attaquants de le détourner pour un profit personnel. 
Pour minimiser ce type de risques, il est recommandé d’effectuer des audits sur les \textit{bridges.} \\

Une faiblesse spécifique des \textit{bridges trusted} repose sur le fait que les utilisateurs doivent léguer le contrôle de leurs \gls{actif}s et faire confiance aux \gls{vérificateur}s externes aux \gls{blockchain}s. Sauf que dans certains cas, ces derniers peuvent coopérer pour tromper les utilisateurs en récupérant leurs \gls{actif}s puis en disparaissant comme dans les \textit{rug pull}\cite{EthereumRisks}. Ce modèle d’escroquerie peut être scindé en deux catégorie : les \textit{hard rug pull} et les \textit{soft rug pull}\cite{Hacken}. Le premier cas est basé sur un piège présent dans le code d’un \textit{\gls{smart contract}} empêchant les utilisateurs d’utiliser ou revendre les \gls{actif}s frappés, seul le fraudeur en a le droit. Il peut donc en toute tranquillité revendre les \gls{actif}s et récupérer l’argent. En revanche, pour les \textit{soft rug pull}, les utilisateurs ne sont pas coincés avec des \gls{actif}s verrouillés mais les fraudeurs utilisent des techniques psychologiques. En effet, les escrocs rendent attirant leur projet pour que les clients investissent et hésitent à se retirer par peur de perdre leur investissent (souvent de taille conséquent) puis les créateurs de la fraude disparaissent avec leurs \gls{actif}s.\\

Comme vu dans la section présentant les différentes méthodes d’échange des \textit{bridges}, ces derniers frappent les \gls{actif}s désirés sur la chaîne destinataire. Certains attaquants peuvent profiter de ce mécanisme de frappe pour effectuer ce qu’on appelle une \textit{Infinite Mint Attack}.\cite{ChainLinkRisks} Cette attaque peut se résumer à un \textit{hacker} générant un nombre élevé d’\gls{actif}s en utilisant une faille d’un \textit{bridge} sans verrouiller ou brûler d’\gls{actif}s sur sa \textit{\gls{blockchain}}. Suite à cela, l’individu réintroduit ces \gls{actif}s sur le marché ce qui fait violemment baisser leur coût ce qui engendre un risque financier systémique.\\

Les \textit{\gls{blockchain} Bridges} sont devenus un outil indispensable des échanges centralisés très rapidement, mais il ne faut pas oublier que ces protocoles sont relativement récents. Créés par de petites \gls{blockchain}s comme Syscoin et NEAR Protocol dans le but de rentre leurs chaînes interopérables avec les applications décentralisées d’\gls{Ethereum}, les premiers bridges datent de 2020\cite{Bitstamp}. Par conséquent, nous ne connaissons pas encore le comportement des \textit{bridges} lorsqu’ils font face à des scénarios sortants de la norme comme des attaques réseaux, un retour en arrière sur les transactions d’une \gls{blockchain} (souvent désigné par le terme \textit{rollback}) ou bien pendant une congestion du réseau. Ces zones d’incertitudes peuvent donc être une source de risques. \\

\begin{figure}[h!]
    \centering
\includegraphics[scale=0.30]{centralisation/imagesBridges/GraphLossesBridges.png}
    {\scriptsize
            Source: \url{https://www.treehouse.finance/insights/blockchain-and-interoperability-globalization-3-0}}
    \caption{Pertes en millions de dollars des bridges les plus connus.}
    \label{fig:GraphBridges}
\end{figure}

\subsection{Le trilemme de l’interopérabilité}

Malgré l’existence de plus d’une centaine de \textit{bridges} différents, les développeurs et les utilisateurs voulant utiliser un \textit{bridge} doivent faire des concessions lors de leur choix vis-à-vis des trois notions de \textit{trustless}, d’extensibilité (\textit{extensible}) et de généralisation (\textit{generalizable}).\\

Le mot trustless peut être traduit par «sans confiance». Si un \textit{bridge} est caractérisé comme \textit{trustless}, cela signifie que celui-ci possède un niveau équivalent à celui d’une ou des chaînes sous-jacentes, il est donc pas nécessaire de faire confiance à une entité externe aux \textit{blockchains}.  La notion  d’extensibilité signifie que le \textit{bridge} est compatible un grand nombre de chaînes.
Un \textit{bridge} respecte la notion de généralisation s’il est capable d'échanger n’importe quel type de données accepté par les deux chaînes.\\

Pour illustrer ces termes, il est possible de les appliquer aux types de vérification appartenant aux \textit{bridges}. La vérification locale respecte les notions d’extensibilité et de \textit{trustless} puisque qu'elle est applicable sur tous les \textit{bridges} peu importe les chaînes reliées et le niveau de fiabilité dépend de la chaîne la plus faible.
La vérification native ne respecte pas la notion d’extensibilité car le \textit{bridge} n’est pas réutilisable. Néanmoins elle respecte la notion de généralisation parce qu’elle est codée de manière spécifique aux \textit{\gls{blockchain}s} reliées au \textit{bridge}. Le critère basé sur la notion \textit{trustless} est également rempli étant donné que le niveau de fiabilité dépend des \gls{vérificateur}s des chaînes.
La vérification externe ne respecte pas la notion de \textit{trustless} cependant elle est fortement extensible et générale vis-à-vis des données. \cite{Ngrave}

 Suite au paragraphe précédent, il est possible de constater que les \textit{bridges} interopérables ne respectent que deux des trois notions énoncées. Ce problème est connu sous le nom de trilemme de l’interopérabilité. 

\subsubsection{Une solution optimiste}

Une solution proposée pour résoudre ce trilemme est un \textit{optimistic bridge}(\textit{bridge} optimiste) nommé vis-à-vis de sa vérification portant le même nom\cite{OptimisticBhuptani}. En effet, contrairement aux \textit{bridges} vérifiés de manière native, locale ou externe, la vérification optimiste dépend de l’utilisation d’une latence lors de la confirmation du transfert des \gls{actif}s entre les \textit{\gls{blockchain}s}. Cela priorise donc la sécurité au détriment de la vivacité, puisque les transactions sont par conséquent plus lentes mais sécurisées car le \textit{bridge} est \textit{trustless}. \\

Voici plus en détails le déroulement du processus de vérification optimiste d’un \textit{bridge}.
Ceci commence par l’envoi d'une demande de transaction de la part d’un utilisateur ou d’une application décentralisée vers la \textit{\gls{blockchain}} native par le biais d’une fonction contrat. Cette demande est ensuite acceptée par un validateur (une entité équivalente à un \gls{vérificateur}, la seule différence étant leur rôle) puis inscrite sur un \textit{block} de la chaîne.\cite{NomadDocsVerification}
Ensuite un \gls{vérificateur} a pour rôle de vérifier la transaction. Pour cela, le \gls{vérificateur} signe le haché des données envoyées précédemment. 
Suite à cela, n’importe quel système de relais peut lire le haché signé sur la chaîne originelle et l’inscrire sur une ou plusieurs chaînes destinataire. Les validateurs de la chaîne destinataire valide et l'inscrivent sur leur chaîne. Cette action déclenche alors une latence de trente minutes pendant laquelle un observateur peut signaler et prouver une fraude effectuée par la chaîne native ce qui déconnectera la communication avec la chaîne destinataire. 

Deux scénarios sont alors possibles. Premier cas, si aucun observateur ne se manifeste, les données de la transaction sont finalisées puis traitées par la chaîne destinataire. Les validateurs sont récompensés pour leur travail avec une partie des frais de transaction. \cite{Fees} Deuxième cas, un observateur prouve une fraude pendant les trente minutes accordées. Le \gls{vérificateur} ayant fraudé est pénalisé par la perte de sa récompense (qui sera obtenu par l'observateur) et son exclusion du réseau.\cite{EthereumSlashing}

\subsubsection{Possibles faiblesses de l’optimisme et leurs solutions}

Le bon fonctionnement des \textit{bridges} optimistes dépend des chaînes auquelles il est rattaché donc tant que ces dernières sont protégées correctement la seule conséquence d’une faille du \textit{bridge} est l’arrêt système plutôt qu’une perte de fonds comme cela peut être le cas avec les autres types de \textit{bridge}.

Les deux acteurs principaux ayant les moyens de nuire au bon fonctionnement du \textit{bridge} ainsi que de la transaction est l’agent \gls{vérificateur} ainsi que l’observateur car ces deux rôles ont de l’influence sur cette dernière. \\

Le premier cas impliquant le \gls{vérificateur} se nommant \textit{Updater Fraud} (fraude du \gls{vérificateur})  fut déjà mentionné lors de la présentation du fonctionnement du \textit{bridge} optimiste. Ce dernier repose sur le fait que toute transaction doit passer par le \gls{vérificateur} et que par conséquent toute fraude est originaire de ce dernier. Sinon si cela venait d’un autre participant, le \gls{vérificateur} n’aurait alors pas accepté la transaction. C’est pourquoi, lors de l’intervention d’un observateur prouvant une fraude, le \gls{vérificateur} est sanctionné par le retrait sur son solde d'un montant équivalent à la récompense promise et par son exclusion du réseau de la \textit{blockchain}. \\

La seconde faiblesse liée aux \gls{vérificateur}s est un \textit{Updater DoS} ou déni de services de la part du \gls{vérificateur}. En effet, il est possible que le processus soit interrompu si un validateur arrête de signer empêchant l'échange inter-chaînes de se produire.
Une solution a été implémentée pour palier à cela comme la mise en place d’un système de substitution avec la présence de plusieurs \gls{vérificateur}s sur une même chaîne afin de pouvoir prendre le relai en cas de manque de réponse de la part de celui étant rattaché au transfert.  Pour éviter que ce scénario se produise fréquemment le \gls{vérificateur} ayant manqué son tour lors de la signature (que cela soit accidentel ou voulu) est pénalisé de la même manière que le cas précédent. \\

Maintenant que les possibles obstacles au bon fonctionnement du \textit{bridge} liés aux \gls{vérificateur}s ont été mis en lumière, il est également possible que l’observateur ait un comportement malveillant.  Effectivement, malgré l’absence de tromperie (puisque le \gls{vérificateur} remplit son rôle), l’observateur peut abuser du mécanisme de déclaration de fraude pour impacter le bon déroulement du procédé. \\

La faculté de l’observateur à pouvoir couper la connexion s’il conteste la transaction lui permet d’effectuer un déni de service appelé \textit{Watcher DoS}. C’est pourquoi il lui ait possible de fermer définitivement la connexion d’une transaction si ce dernier continue sans cesse de couper le processus sans raison valable. Heureusement, la fermeture ne concerne que la connexion et n’impacte en aucun cas le système du \textit{bridge}. Cependant cette attaque semble irrationnelle en terme de ressources et de temps car l’observateur effectuant le déni de service ne gagne rien financièrement contrairement au processus habituel. En effet, si un observateur prouve une fraude correctement, ce dernier peut récupérer la récompense du \gls{vérificateur}. Mais ici puisqu’aucune fraude n’est prouvée les données se trouvant sur la chaîne d’origine sont conservées et sécurisés. Cela cause seulement une perte de temps pour l’utilisateur ou l’application décentralisée voulant effectuer l'échange d’une \textit{\gls{blockchain}} à une autre.

Une réponse à ce problème actuellement mise en œuvre par le \textit{bridge} de \gls{Nomad} est la présence d’un groupe restreint d’observateurs autorisés à contester, de cette manière il est facile de connaître les observateurs malveillants. Chaque observateur possède une clé permettant de signer une attestation confirmant la présence d’une fraude dans la transaction, chaque \textit{bridge} stocke un ensemble contenant les adresses des attestations appartenant aux observateurs autorisés. Si l’attestation reçue par le \textit{bridge} est présente dans l’ensemble alors la connexion est rompue\cite{NomadDocsWatcher}.
Sur le long terme, une proposition consistant à la mise en place de frais si l’on souhaite contester est en train d’être étudiée. Le montant doit répondre à deux contraintes: ce dernier doit être assez haut pour dissuader les observateurs malhonnêtes mais assez bas pour que ceux ayant réellement l’envie de prouver de manière valide une fraude existante puissent le faire. Dans la continuité de cette solution, il serait également possible de récupérer la signature de la déconnexion générée par l’observateur sur la chaîne originale et de le pénaliser en lui retirant les frais qu’il a payé tel une garantie\cite{OptimisticBhuptani}.



\subsection{Wormhole}
En 2017, une cryptomonnaie adossée à la \textit{blockchain} Solana a émergée avec des caractéristiques
similaires à Ethereum : \textit{blockchain} publique, \textit{smart contracts}.\\
Solana est devenue de facto une \textit{blockchain} concurrente à Ethereum et est aujourd'hui 
la onzième \textit{blockchain} en terme de capitalisation selon l'aggrégateur de marché Coinmarketcap.\\
Un besoin d'échanger des actifs entre les \textit{blockchains} Ethereum et Solana est apparu, 
d'où l'introduction en 2020 de la première version de Wormhole.
Initialement, Wormhole v1 a été concu comme un \textit{bridge} entre Ethereum et Solana.
Depuis, Wormhole s'est développé au-delà de Solana avec le lancement d'une deuxième version en 2021 
en tant que protocole générique de passage de messages.\\
À l'écriture de ce rapport, 22 \cite{wormholeNetwork} \textit{blockchains} sont compatibles avec Wormhole 
dont : BNBChain, Ethereum, Moonbeam, Polygon, Solana...\\
Le protocole émet un message à partir d'une \textit{blockchain} source qui est validé par un réseau de 
gardiens.\\ 
Le message est ensuite envoyé à la \textit{blockchain} cible pour être traité.

\subsubsection{VAA (\textit{Verified action approval})}

Lorsqu'un \textit{smart contract} envoie un message \textit{crosschain} comme un verrouillage
de jetons sur une \textit{blockchain} source et une demande de frappe de jetons sur une 
\textit{blockchain} cible, celui-ci interargit avec un \textit{core contract} \cite{wormholeCoreContract}.
Un \textit{core contract} est déployé sur toutes les \textit{blockchains} compatibles avec le protocole 
Wormhole. Tout \textit{core contract} est observé par le réseau de gardiens.
Un message Wormhole est émis grâce à la fonction \textit{publishMessage()} prenant en entrée le \textit{payload}.
La sortie de cette fonction est un \textit{sequence number}, un numéro d'index unique pour le message.
Combiné à l'adresse du contrat de l'émetteur et à l'identifiant de la chaîne de l'émetteur, le message 
correspondant peut être récupéré auprès d'un nœud du réseau de gardiens.\\
Un message Wormhole est vérifié grâce à la fonction \textit{parseAndVerifyVAA()} prenant en entrée le message.
Selon la validité de l'entrée, la fonction retourne en sortie le \textit{payload} ou une exception.
\newpage

VAA \cite{wormholeVAA} est la primitive de messagerie de base de Wormhole. Un VAA contient une en-tête 
ainsi qu'un \textit{body}. L'en-tête contient l'index des gardiens ayant signés le message et la collection des signatures.
L'en-tête permet au \textit{core contract} de vérifier l'authenticité du VAA.
Quant au \textit{body}, il contient des informations comme le numéro d'identification de la chaîne 
Wormhole du contrat émetteur, l'adresse du contrat émetteur, le \textit{sequence number} 
et le \textit{payload}.\\ 5 \textit{payloads} peuvent être utilisés dont \textit{Transfer} et 
\textit{AssetMeta}, attestant les méta-données du jeton.\\
Le \textit{payload AssetMeta} est obligatoire avant un premier transfert.
En effet, le \textit{payload Transfer} n'informe pas la chaîne B des meta-données du jeton verrouillé.
En l'absence de connaissance de ces informations, il n'est pas possible pour la \textit{blockchain} B 
de frapper la quantité correcte de jetons.\\
Si l'on souhaite ensuite transférer des jetons depuis une \textit{blockchain} A vers une 
\textit{blockchain} B, il faut verrouiller les jetons sur A et les frapper sur B.
D'où l'utilisation du \textit{payload Transfer} contenant des informations comme la 
quantité de jetons transférés, l'adresse de la chaîne d'origine et de destination, 
le numéro d'identification de la chaîne d'origine et de destination..
Une preuve doit être fournie que les jetons sur A sont verrouillés avant que la frappe puisse 
avoir lieu sur B. La signature des gardiens sur le VAA correspondant est la preuve apportée à 
la \textit{blockchain} B que le verrouillage a bien été effectué et que la frappe de jetons sur 
B est légitime.

\subsubsection{Gardiens}

Un gardien \cite{wormholeGuardian} est une autorité de confiance qui a comme de rôle valider 
(par une signature) le \textit{payload} contenu dans un VAA.
Comme évoqué précédemment, le réseau de gardiens observe tous les messages \textit{crosschain} via la 
surveillance des \textit{core contracts}.
Le réseau de gardiens est composé de 19 gardiens à parts égales sans chef (\textit{leaderless}).
Il est conçu pour servir d'oracle à Wormhole et est l'élement le plus critique de l'écosystème.
Si une majorité de deux tiers ou plus des gardiens signent le même VAA, alors le consensus est atteint : 
le VAA est automatiquement considéré valide par  tous les contrats Wormhole sur toutes les 
\textit{blockchains} et le \textit{payload} est actionné. 
Chaque gardien utilise un algorithme de signature à courbe elliptique : ECSDA pour 
\textit{Elliptic Curve Signature Digital Algorithm}.
Plus précisément, chaque gardien se réfère à «secp256k1» comme paramètres de la courbe elliptique, 
aussi utilisé par les \textit{blockchains} Bitcoin et Ethereum.\\
Le modèle de consensus utilisé est une \textit{Proof of Authority} (PoA) avec un système de 
\textit{multisignature} M/N \cite{wormholeChainswap}, c'est à dire que M clefs parmi N sont nécessaires 
pour signer un VAA. Ce modèle permet un traitement rapide des transactions et une dispense de participation monétaire, par rapport à la preuve de travail (PoW) et la preuve 
de participation (PoS). Cependant, il présente également des désavantages : le système est par 
\textit{design} centralisé et dépend d'un petit groupe de nœuds pouvant créer un point de 
défaillance unique par l'utilisation commune d'une fonction vulnérable. Il est questionnable de restaurer des tiers de confiance dans le cadre d'un système 
devenu populaire grâce à l'absence de tels autorités. Wormhole justifie la décentralisation de leur 
système \cite{wormholeGuardian} par la présence de plusieurs parties (et non d'un seul) dans le contrôle du réseau.
Selon notre analyse, la décentralisation résulte de l'absence d'un ou plusieurs tier(s) de confiance lorsque deux parties 
souhaitent réaliser une transaction.
\newpage

\subsubsection{Relais}

Un relai \cite{wormholeRelayer} est un processus qui délivre un ou plusieurs VAA(s) à une destination.
Les relais ne sont ni de confiance, ni privilégiés, ils écoutent directement le réseau de gardiens 
via l'intermédiaire d'un processus espion. Ces relais ne peuvent pas compromettre l'intégrité d'un VAA 
car une altération serait détectée lors du processus de vérification des signatures. Cependant, il n'est 
pas assuré qu'un relai transmette un VAA à destination, d'où une perte de disponibilité. Il est conseillé
d'héberger soi-même ces relais pour supporter son application.

\begin{figure}[h!]
    \centering
    \includegraphics[scale=0.5]{centralisation/uml_design_v2.png}
    \label{fig:wormholeDesign}
    \caption{Architecture Wormhole \cite{wormholeArch}}
  \end{figure}

% @startuml
% rectangle r1 as "Source Token Bridge 
% Relayer Contract"
% rectangle r2 as "Source Token 
% Bridge"
% rectangle r3 as "Source Core 
% Contract"
% storage "Guardians" as r4
% hexagon r5 as "Off-chain 
% Message Relayer" 
% rectangle r6 as "Target Token Bridge 
% Relayer Contract"
% rectangle r7 as "Target Token 
% Bridge"
% rectangle r8 as "Target  Core 
% Contract"

% r1 -> r2 : Transfer()
% r2 -> r3 : publishMessage()
% r3 -do-> r4 : Guardien reads message
% r4 -left-> r5 : Signed message

% r5 -do-> r6 : Signed message
% r6 -> r5 : Relayer fee

% r6 -> r7 : Signed message
% r7 -> r6 : Wrapped token

% r7 -> r8 : parseAndVerifyVAA()
% @enduml




\subsection{Analyse d'attaques}
% Auteur Romain TESTUD
\subsubsection{Mise en contexte}
Les \textit{\gls{blockchain}s} et leurs protocoles d'échanges ne sont pas exemptes d'attaques informatiques ou bien de défaillances.
Ces attaques peuvent cibler des portefeuilles (attaques sur des \textit{hot wallets}\footnote{portefeuille de cryptomonnaies en ligne, à différencier des \textit{Cold Wallets}, des portefeuilles hors lignes}) ou encore des \textit{bridges}. 
Ce sont ces dernières qui nous ont intéressées dans le cadre de ce projet de recherche sur les échanges inter-\gls{blockchain}s. 
Les bridges, comme explicité dans la partie dédiée du rapport, sont des protocoles permettant la circulation de données entre deux \textit{\gls{blockchain}s} différentes.\\
Nous avons, au cours de nos recherches, trouvés de nombreux cas d'attaques sur des protocoles d'échanges inter-\gls{blockchain}s. 
De manière à illustrer les types d'attaques possibles et les points critiques de ces systèmes nous allons décrire deux attaques parmi les plus importantes : \textit{Wormhole} et \textit{Nomad}.

\subsubsection{Le cas Wormhole}
Nous vous avons présenté le protocole \textit{Wormhole} dans la partie précédente. 
Le 2 Février 2022, une attaque exploite une erreur d'implémentation dans une \textit{\gls{dApp}} sur la chaîne Solana \cite{SolMed} \cite{SolRekt}. 
Pour se faire l'attaquant à réussi à contourner la vérification des signatures des gardiens en exploitant
une correction de bug ayant été publié sur le code source du projet mais n'étant pas encore effective en production.
Ainsi il à réussi à récupérer l'équivalent de 120 000 \textit{ETH} en \textit{whETH} (\textit{Wormhole ETH}). 
Lors d'un transfert de jetons d'une chaîne à une autre, plusieurs étapes sont réalisées par différentes fonctions.
Après la formulation de la transaction, une fonction va se charger de récupérer les signatures des gardiens dans un \textit{SignatureSet}\footnote{Ensemble de signatures de gardiens}, ces dernières sont ensuite vérifiées. 
Pour cela, une fonction nommée \texttt{verify\_signature} va appeler un programme de vérification de Solana permettant l'analyse du \textit{SignatureSet}. 
L'appel à ce programme se fait de la manière suivante, en utilisant le nom \texttt{sysvarinstruction} \cite{SolGitError} dans la transaction. 
Dès lors que les signatures sont validées, un \textit{VAA} peut être émis et transmis vers la \textit{\gls{blockchain}} souhaitée. \\
La transaction de l'attaquant étant frauduleuse, il n'aurait donc pas pu obtenir de signatures des gardiens. 
Pour contourner cette étape de récupération des signatures la transaction de l'attaquant était dotée d'un \textit{SignatureSet} correspondant à une transaction antérieure. 
Seulement, n'étant pas pour la bonne opération cet ensemble ne peut pas être approuvé par \texttt{verify\_signature}. 
C'est ici que l'attaquant à utilisé un défaut d'implémentation pour valider son \textit{SignatureSet}. 
Comme décrit précédemment, la fonction \texttt{verify\_signature} appelle un programme pour effectuer la vérification des signatures. 
Cependant il n'y à pas de vérification faites sur le programme appelé, l'attaquant à pu donc utiliser une adresse différente lui permettant de valider sa transaction. 
Avec le  \texttt{SignatureSet} ainsi validé, l'attaquant a pu générer un \textit{VAA} valide et pu déclencher une frappe de jeton vers son propre compte sans en avoir déposé au préalable. 
La correction de cette faille était contenue dans la mise à jour évoquée en début de paragraphe\cite{SolGitFixed}, permettant la vérification du programme appelé pour la vérification. 

\subsubsection{Le cas Nomad}
Nomad est un protocole d'interopérabilité entre chaînes permettant de passer des \gls{actif}s entre deux \textit{\gls{blockchain}s} différentes. 
Pour fonctionner, ce protocole fait appel à des applications décentralisées opérant sur les chaînes du réseau. 
Une première \textit{\gls{dApp}} appelée \textit{réplica} est déployée sur les \textit{\gls{blockchain}s} recevant les messages, elle fait office de "boite de réception". 
Une seconde \textit{\gls{dApp}} appelée \textit{home} est déployé sur les \textit{\gls{blockchain}s} émettrices de message. \\
Le 1\textsuperscript{er} août 2022 une attaque exploitant une erreur d'implémentation sur l'application \textit{Réplica} a engendré une perte de 190 millions de dollars en liquidité \cite{NomadMedium} \cite{NomadRekt}.
Cette attaque s'est déroulée après le déploiement d'une mise à jour, un moyen de contourner la vérification des signatures du message étant apparu. 
En analysant l'application \textit{Réplica} après la mise à jour, nous pouvons voir que lors d'une initialisation, la racine des messages, appelée \texttt{\_commitedRoot}, est initialisée à $0$, ce signifiant que le message n'a pas encore été validé. 
\begin{lstlisting}[caption={Fonction \textit{initialize} de \textit{Réplica} contenant une erreur \cite{NomadGitError}}]
    function initialize(
        uint32 _remoteDomain,
        address _updater,
        bytes32 _committedRoot,
        uint256 _optimisticSeconds
    ) public initializer {
        __NomadBase_initialize(_updater);
        // set storage variables
        entered = 1;
        remoteDomain = _remoteDomain;
        committedRoot = _committedRoot;
        // pre-approve the committed root.
        confirmAt[_committedRoot] = 1;
        _setOptimisticTimeout(_optimisticSeconds);
    }
\end{lstlisting}

Dans les lignes précédentes nous observons cette affectation : \texttt{confirmAt[\_commitedRoot] = 1}, le rôle de cette ligne est de pré-approuver la racine d'un message. 
Cette fonction est utilisée pour approuver le premier message lors du déploiement du contrat sur une \textit{\gls{blockchain}}. 
Or ici, la valeur de la racine à été initialisée a $0$, donc cette racine devient une racine valide pour la fonction de vérification des messages. 
Seulement comme nous l'avons énoncé précédemment, $0$ est la valeur par défaut pour un message n'ayant pas encore été vérifié. 
Ainsi, lors de l'émission d'un message par la fonction \texttt{process}, tout message non vérifié sera envoyé. 
Cette erreur d'implémentation a permis à des pirates d'effectuer plusieurs transactions frauduleuses et de retirer l'équivalent de 190 Millions de dollars dans la réserve de liquidité du bridge de Nomad. 
Le contrat à été corrigé, dans une mise en ligne datant du 3 Septembre 2022, tel que la racine $0$ n'est plus pré-approuvée. 

\begin{lstlisting}[caption={Fonction corrigée de l'application \textit{Réplica} \cite{NomadGitFixed}}]
    function initialize(
        uint32 _remoteDomain,
        address _updater,
        bytes32 _committedRoot,
        uint256 _optimisticSeconds
    ) public initializer {
        __NomadBase_initialize(_updater);
        // set storage variables
        entered = 1;
        remoteDomain = _remoteDomain;
        committedRoot = _committedRoot;
        // pre-approve the committed root.
        if (_committedRoot != bytes32(0)) confirmAt[_committedRoot] = 1;
        _setOptimisticTimeout(_optimisticSeconds);
    }
\end{lstlisting}


%\subsubsection{DeFi hacklabs}
%Lors de nos recherches sur des attaques sur des protocoles inter-\gls{blockchain}s, nous avons découvert \textit{Web3sec}, un groupe de recherche centré sur la sécurité du web3. 
%Le groupe met à disposition des ressources indexés sur une page notion (en annexe) :
%\begin{itemize}
%    \item Plusieurs dépots \textit{Github} pour étudier les attaques et apprendre les vulnérabilités sur ces types de programmes.
%    \item \textit{DeFi Hacks Analysis - Root Cause} : Une base de données d'analyses d'attaques sur des solutions et organismes traitant sur des \gls{blockchain}s, les analyses sont sourcées et redirigent vers le dépot \textit{GitHub} de reproduction des attaques.
%    \item Un ensemble d'outils utilitaires tels que des outils de debug de transaction, des dashboards ou encore des newletters.
%\end{itemize}
%Ces outils nous ont permis d'explorer de nombreuses sources concernant les attaques sur les protocoles inter\gls{blockchain}s.

\subsubsection{Contre-mesures et solutions envisageables}
Comme nous avons pu le voir, de nombreux cas d'attaques sont observables sur des protocoles d'échanges centralisés. 
Dans la plupart des cas, elles résultent de problèmes d'implémentations et autres oublis dans les codes sources des protocoles utilisés. 
Cela peut être expliqué par le fait que la \textit{\gls{blockchain}} est un domaine qui évolue très vite et chaque innovation technique peut rapporter des parts de marché importantes au premier arrivé. 
De plus, de nombreux acteurs se spécialisent dans ce domaine sans nécessairement avoir une grande culture de la cybersécurité.
Il se peut donc que des erreurs d'implémentations paraissant évidentes ne soient pas relevés lors de la mise en production. \\ 
Des moyens de limiter le plus possible l'apparition de tels événements sont néanmoins possibles. 
Tout d'abord des standards de sécurité pourraient être mis en place afin de déterminer un socle minimal à atteindre. 
Des audits et des analyses de sécurité peuvent être mis en place pendant la production ou après la publication des protocoles, en respectant un cycle de développement "classique".
%Nous pouvons aussi nous questionner quant à l'implication du tier de confiance dans l'apparition de ces failles. 
%En effet, les attaques les plus importantes ont comme point d'entrée une faille dans la structure de l'intermédiaire. 
%Cette menace constante d'attaques a mené à des recherches visant à soutirer ce tiers des échanges inter-\gls{blockchain}. Ce menant aux échanges décentralisés.


\subsection{Les limites du centralisé}
% Auteur : Romain TESTUD
\begin{frame}{Les limites du centralisé}
    \begin{itemize}
        \item Manque de transparence.
        \item Manque de maturité en matière de sécurité.
        \item Questionnement sur le tiers de confiance
    \end{itemize}
    \pause
    \begin{block}{Le tiers de confiance}
        \begin{itemize}
            \item Source d'attaques.
            \item Point critique des CEX.
        \end{itemize}
        $\rightarrow$ Volonté de se passer d'un tiers de confiance.
    \end{block}
\end{frame}
\section{Conclusion}
\subsection{Les Plate-formes d'échanges centralisés}
\subsubsection{Définition}
Nous avons commencé nos recherches en nous intéressant en premier lieu aux moyens d'échanges les plus répandus. 
Cela nous a mené vers les plate-formes d'échanges centralisé. 
Ce sont des plate-formes, pouvant prendre la forme d'applications web, qui permettent aux utilisateurs d'acheter, de vendre ou d'échanger des \gls{actif}s numériques contre d'autres \gls{actif}s numériques ou en monnaies fiduciaires. 
Ces plate-formes peuvent opérer sur des \textit{\gls{blockchain}s} publiques ou être dédiées à une utilisation en interne. 
Elles sont dites centralisées car elles sont gérées par une entreprise ou une organisation hiérarchisée qui contrôle les transactions et les fonds des utilisateurs.
Ces plate-formes sont donc considérées comme des tiers de confiance et agissent en tant qu'intermédiaires entre les acheteurs et les vendeurs en assurant la sécurité, la liquidité et la rapidité des transactions.
C'est la solution la plus utilisée dans le secteur des \gls{actif}s numériques, elles offrent très souvent une certaine variété de services tels que le prêt et/ou le \textit{stacking} \footnote{Stacking : Action de verrouiller des jetons en vue de recevoir des récompenses \cite{defStack}}.
Elles proposent également un large éventail de cryptomonnaies disponibles.

\subsubsection{Inconvénients et risques}
Nous avons pu tout de même relever certains inconvénients et certains risques pour les utilisateurs liés à l'utilisation de ces plate-formes. 
Tout d'abords, les utilisateurs doivent confier leurs fonds et leurs données à un tier en qui ils doivent avoir confiance. 
Cela peut exposer les utilisateurs à de la fraude, du vol ou encore du piratage si les plate-formes présentent des failles de sécurité. \\
Ensuite, ces plate-formes peuvent être victimes de pannes ou de saturation du réseau pouvant entraîner des retards, des pertes de transactions ou encore du déni de service bloquant ainsi l'accès aux \gls{actif}s des utilisateurs. 
Finalement, ces plate-formes sont soumises à la réglementation et à la surveillance des autorités financières, limitant leur accessibilité dans certains pays ou régions. 
Ce point signifie aussi que les \gls{actif}s de l'utilisateurs sont traçables par les autorités. 


\subsubsection{Fonctionnement}
Ces plate-formes fonctionnent sur le principe de l'\textit{order book method} (méthode du carnet d'ordre\cite{orderBook}), une modélisation des ordres d'achats et de vente des jetons.
Un ordre étant une demande d'un utilisateur visant à réaliser une opération à un prix et une quantité donnée. 
Cette méthode comprend deux parties: l'offre et la demande. L'offre regroupe les ordres d'achats émis par des utilisateurs sur la plate-forme et la demande, les ordres de vente.
Lors d'un dépôt, l'utilisateur s'étant au préalable enregistré sur la plate-forme, il va déposer les fonds souhaités dans un porte monnaie. 
La plate-forme va ensuite crée un \textit{IOU}\footnote{I Owe You, c'est la dette de la plate-forme envers l'utilisateur permettant de bloquer la valeur de la monnaie déposée par l'utilisateur\cite{IOU}} 
ce dernier sera échangé contre le crypto-\gls{actif} souhaité lors d'un échange ou d'une vente. \\ 
\begin{figure}[h!]
    \centering
    \includegraphics[scale=0.5]{centralisation/echange.png}
    \label{fig:simplifiedcex}
    \caption{Échange d'un jeton en Euros}
\end{figure}
Dans le cadre des échanges inter-\gls{blockchain}s, les plate-formes d'échanges utilisent des \textit{bridges} reliant les différentes \textit{\gls{blockchain}s}. 
Ces protocoles seront explicités dans la partie suivante.
Cependant, nous n'avons pas pu trouver de plus amples explications quant aux fonctionnements des plate-formes, notamment les protocoles précis utilisés lors des échanges. 
Les documentations disponibles pour les plateformes d'échanges étant à destination des utilisateurs finaux. 




\subsection{Les Blockchain Bridges}
%Auteure : Eloïse Rotondo

\subsubsection{Fonctionnement}

Comme son nom l’indique, un \textit{\gls{blockchain} bridge} également appelé \textit{\gls{cross-chain} bridge} est un protocole reliant deux \textit{\gls{blockchain}s} entre elles de manière unilatérale ou bilatérale dans une optique d’interopérabilité.\\

Dans la but de comprendre la popularité des \textit{bridges} en tant que protocole d’interopéralité, il faut en premier lieu s’intéresser au marché de la cryptomonnaie. Actuellement \gls{Bitcoin} domine en représentant 40,5\% de ce dernier, suivie ensuite par \gls{Ethereum} avec 19,5\%. Cela laisse donc 40\% du marché formé de nombreuses crytomonnaies plus petites et plus indépendantes. C’est donc naturellement, qu’une forte demande de possibilité d’échanges entre les \textit{\gls{blockchain}s} ait vu le jour de la part des utilisateurs ayant plusieurs cryptomonnaies\cite{Ngrave}.\\

Il existe trois différentes manières de déplacer les \gls{actif}s en tant que \textit{bridge}. Tout d’abord, le mécanisme de \textit{Lock and Mint} signifiant Verrouiller et Frapper, les \gls{actif}s se trouvant sur la chaîne de départ sont verrouillés sur celle-ci pour être ensuite créés sur la chaîne destinataire. Un autre mode d'échange est celui du \textit{Burnt and Mint}, ce dernier est très similaire à celui déjà présenté, la seule différence étant que les \gls{actif}s sont directement effacés plutôt que verrouillés. Pour finir, les échanges atomiques entre chaînes (Atomic Swaps) permettent un échange direct en pair-à-pair d'\gls{actif}s entre la chaîne d’origine et la chaîne de destinataire.\cite{EthereumMechanism}

\subsubsection{Mécanisme de vérification}

Comme évoqué précédemment, deux \textit{\gls{blockchain}s} ne peuvent pas communiquer directement entre elles, par conséquent lors de l’utilisation d’un \textit{bridge} les deux chaînes ne se connaissent pas et ont seulement connaissance des évènements se produisant sur leur chaîne respectives. Il est donc nécessaire d’établir une relation de confiance entre les deux chaînes pour qu’elles puissent accepter de communiquer. Pour cela, les \textit{bridges} emploient un mécanisme utilisant des \gls{vérificateur}s. Un \gls{vérificateur} est une entité connectée en tant que \gls{noeud} au réseau de la \textit{blockchain}. Ce dernier agit comme autorité de confiance, vérifiant et validant les transactions sur cette dernière. Un noeud d'une \textit{blockchain} est un ordinateur connecté au réseau de cette dernière. Un \textit{client} est un logiciel permettant de transformer un ordinateur en noeud. \cite{EthereumNodeClient} \\


Il existe un grand nombre de \textit{bridges}, chacun avec leurs propres spécificités mais ils peuvent généralement être séparés en deux catégories les \textit{Trusted \gls{blockchain} Bridge} et les \textit{Trustless \gls{blockchain} Bridge}. \\


Les \textit{Trusted Bridges} sont vérifiés de manière externe car ils utilisent un ensemble de \gls{vérificateur}s tiers pour transmettre des données entre les chaînes. Ils ont pour avantage leur rapidité, leur moindre coût et la facilité d'échange avec tous les types de données acceptés par les \textit{\gls{blockchain}s}. Cependant les \gls{vérificateur}s externes sont moins fiables que ceux de la chaîne.\cite{EthereumBridges}

Les \textit{Trustless Bridges} sont désignés comme \textit{trustless} car ils dépendent des chaînes dont ils font l’intermédiaire pour transférer des données ou des \gls{actif}s. Par conséquent leur niveau de fiabilité est égal à celui des \textit{\gls{blockchain}s} et il n’est pas nécessaire de faire confiance à un ensemble de \gls{vérificateur}s tiers (contrairement aux autres \textit{bridges}). Pour cette raison, ils sont reconnus comme étant plus fiables que les \textit{trusted bridges}.\cite{EthereumBridges}\\


 Les \textit{bridges} peuvent également être distingués en fonction de leur type de vérification. Les plus connus sont la vérification native, externe et locale.\cite{InteroperabilityBhuptani} \\

La vérification native commence par l’utilisation d’un \gls{noeud léger}. \cite{NomadDocsNative} Un noeud léger aussi connu sous le nom de client léger est un logiciel permettant de connecter les noeuds des \textit{\gls{blockchain}s} entre elles. Il est codé comme un \textit{\gls{smart contract}} puis est employé de la chaîne de l'expéditeur vers la machine virtuelle de la chaîne destinataire. Si les \gls{vérificateur}s de données de la chaîne de l'expéditeur agissent de manière correcte alors le noeud léger est vu comme véridique par la chaîne réceptionnant les \gls{actif}s ou données et peut être utilisé de manière bilatérale.
 Un avantage de cette solution est qu’elle est reconnue comme étant celle reposant le moins sur la confiance parmi celles existantes car les chaînes ne se fient qu’à leurs propres \gls{vérificateur}s pour effectuer le \textit{bridge}. Un autre bénéfice de ce mécanisme est le fait qu’il n’utilise pas de \gls{vérificateur}s tiers entre les deux \textit{\gls{blockchain}s} et donc la sécuté du réseau dépend des \textit{\gls{blockchain}s} elles-même (ce qui est avantageux car elles sont robutes et préparées aux attaques comme la chaîne d’Ethereum par exemple).
 Un désavantage de cette méthode est que le noeud léger doit être adapté aux consensus des chaînes auquelles il est attaché ce qui le rend inutilisable avec des chaînes différentes. Le noeud léger nécessite également de la maintenance en cas de changement des règles consensus (utilisées pour valider les transactions). Un autre inconvénient découlant du fait que le noeud léger est programmé de manière spécifique est que ce dernier n’est donc pas réutilisable. \\

\pagebreak

La vérification externe consiste en un ensemble de \gls{vérificateur}s n’appartenant pas aux \textit{\gls{blockchain}s} relayant les données entre les deux extrémités du \textit{bridge}. Pour se faire, un certains nombre de \gls{vérificateur}s doivent signer un message provenant de la chaîne d’envoi pour que le chaîne destinataire le reconnaisse comme valide. Par exemple, pour le \textit{bridge} \gls{Wormhole} 13 \gls{vérificateur}s sur 19 doivent avoir signé\cite{NomadDocsExternal}. Ce concept est une primitive cryptographique (algorithme cryptographique de bas niveau servant de base à un système de sécurité informatique) nommée le système de signature à seuil (désignée par TSS pour \textit{Threshold Signature Scheme})\cite{BinanceTSS}. 
Contrairement à la vérifications native, les \textit{bridges} vérifiés de manière externe sont faciles à développer, peuvent être réutilisés sans problèmes et leur maintenance coûte peu. Le désavantage conséquent de cette méthode est que la sécurité dépend des \gls{vérificateur}s tiers du pont ce qui peut fragiliser le système car ils sont généralement moins sécurisés que ceux des \textit{\gls{blockchain}s}. \\


\begin{figure}[h!]
    \centering
\stackunder{
\includegraphics[scale=0.60]{centralisation/imagesBridges/LightClient.png}}
    {\scriptsize
            Source: \url{https://docs.nomad.xyz/the-nomad-protocol/verification-mechanisms/native-verification}}
    \caption{Mécanisme utilisant un noeud léger.}
    \label{fig:LightClient}
\end{figure}

\pagebreak

La vérification externe consiste en un ensemble de \gls{vérificateur}s n’appartenant pas aux \textit{\gls{blockchain}s} relayant les données entre les deux extrémités du \textit{bridge}. Pour se faire, un certains nombre de \gls{vérificateur}s doivent signer un message provenant de la chaîne d’envoi pour que le chaîne destinataire le reconnaisse comme valide. Par exemple, pour le \textit{bridge} Wormhole 13 \gls{vérificateur}s sur 19 doivent avoir signé\cite{NomadDocsExternal}. Ce concept est une primitive cryptographique (algorithme cryptographique de bas niveau servant de base à un système de sécurité informatique) nommée le système de signature à seuil (désignée par TSS pour \textit{Threshold Signature Scheme})\cite{BinanceTSS}. 
Contrairement à la vérifications native, les \textit{bridges} vérifiés de manière externe sont faciles à développer, peuvent être réutilisés sans problèmes et leur maintenance coûte peu. Le désavantage conséquent de cette méthode est que le bon fonctionnement du système dépend des \gls{vérificateur}s tiers ce qui peut le fragiliser car ils sont généralement moins fiables que ceux des \textit{\gls{blockchain}s}. \\

Il est intéressant de noter que les \textit{\gls{blockchain}s} ont également leur propre ensemble de \gls{vérificateur}s sous la forme de \gls{vérificateur} de données. Ces derniers sont utilisés lors de la vérification locale. Lors de la vérification locale, les chaînes se vérifient entre elles en utilisant un \gls{vérificateur} en tant que représentant.

\begin{figure}[h]
    \centering
\includegraphics[scale=0.60]{centralisation/imagesBridges/DiagrammeResumeVerif.png}
\caption{Résumé des types de vérification. (locale, native et externe)}
\label{fig:LocaleVerif}
\end{figure}

\subsubsection{Les risques liés aux Bridges}

La popularité des \textit{\gls{blockchain} Bridges} pour les échanges centralisés ne cesse d’augmenter au fils du temps mais il est important de comprendre que comme tout outil, ces derniers ne sont pas sans risques. \\

Les \textit{bridges trustless} utilisent des \textit{\gls{smart contract}s} lors du processus d’échange dans le but de le rendre autonome afin de ne pas utiliser une entité centrale entre les deux \gls{blockchain}s. Cependant ces \textit{bridges} sont néanmoins centralisés car ils utilisent les \gls{vérificateur}s pour obtenir un consensus lors des transactions.
Un \textit{\gls{smart contract}} étant un script écrit par un développeur, il est possible que certaines erreurs puissent s’être glissées dans le code par inadvertance ou bien qu’il existe des failles dans le programme permettant aux attaquants de le détourner pour un profit personnel. 
Pour minimiser ce type de risques, il est recommandé d’effectuer des audits sur les \textit{bridges.} \\

Une faiblesse spécifique des \textit{bridges trusted} repose sur le fait que les utilisateurs doivent léguer le contrôle de leurs \gls{actif}s et faire confiance aux \gls{vérificateur}s externes aux \gls{blockchain}s. Sauf que dans certains cas, ces derniers peuvent coopérer pour tromper les utilisateurs en récupérant leurs \gls{actif}s puis en disparaissant comme dans les \textit{rug pull}\cite{EthereumRisks}. Ce modèle d’escroquerie peut être scindé en deux catégorie : les \textit{hard rug pull} et les \textit{soft rug pull}\cite{Hacken}. Le premier cas est basé sur un piège présent dans le code d’un \textit{\gls{smart contract}} empêchant les utilisateurs d’utiliser ou revendre les \gls{actif}s frappés, seul le fraudeur en a le droit. Il peut donc en toute tranquillité revendre les \gls{actif}s et récupérer l’argent. En revanche, pour les \textit{soft rug pull}, les utilisateurs ne sont pas coincés avec des \gls{actif}s verrouillés mais les fraudeurs utilisent des techniques psychologiques. En effet, les escrocs rendent attirant leur projet pour que les clients investissent et hésitent à se retirer par peur de perdre leur investissent (souvent de taille conséquent) puis les créateurs de la fraude disparaissent avec leurs \gls{actif}s.\\

Comme vu dans la section présentant les différentes méthodes d’échange des \textit{bridges}, ces derniers frappent les \gls{actif}s désirés sur la chaîne destinataire. Certains attaquants peuvent profiter de ce mécanisme de frappe pour effectuer ce qu’on appelle une \textit{Infinite Mint Attack}.\cite{ChainLinkRisks} Cette attaque peut se résumer à un \textit{hacker} générant un nombre élevé d’\gls{actif}s en utilisant une faille d’un \textit{bridge} sans verrouiller ou brûler d’\gls{actif}s sur sa \textit{\gls{blockchain}}. Suite à cela, l’individu réintroduit ces \gls{actif}s sur le marché ce qui fait violemment baisser leur coût ce qui engendre un risque financier systémique.\\

Les \textit{\gls{blockchain} Bridges} sont devenus un outil indispensable des échanges centralisés très rapidement, mais il ne faut pas oublier que ces protocoles sont relativement récents. Créés par de petites \gls{blockchain}s comme Syscoin et NEAR Protocol dans le but de rentre leurs chaînes interopérables avec les applications décentralisées d’\gls{Ethereum}, les premiers bridges datent de 2020\cite{Bitstamp}. Par conséquent, nous ne connaissons pas encore le comportement des \textit{bridges} lorsqu’ils font face à des scénarios sortants de la norme comme des attaques réseaux, un retour en arrière sur les transactions d’une \gls{blockchain} (souvent désigné par le terme \textit{rollback}) ou bien pendant une congestion du réseau. Ces zones d’incertitudes peuvent donc être une source de risques. \\

\begin{figure}[h!]
    \centering
\includegraphics[scale=0.30]{centralisation/imagesBridges/GraphLossesBridges.png}
    {\scriptsize
            Source: \url{https://www.treehouse.finance/insights/blockchain-and-interoperability-globalization-3-0}}
    \caption{Pertes en millions de dollars des bridges les plus connus.}
    \label{fig:GraphBridges}
\end{figure}

\subsection{Le trilemme de l’interopérabilité}

Malgré l’existence de plus d’une centaine de \textit{bridges} différents, les développeurs et les utilisateurs voulant utiliser un \textit{bridge} doivent faire des concessions lors de leur choix vis-à-vis des trois notions de \textit{trustless}, d’extensibilité (\textit{extensible}) et de généralisation (\textit{generalizable}).\\

Le mot trustless peut être traduit par «sans confiance». Si un \textit{bridge} est caractérisé comme \textit{trustless}, cela signifie que celui-ci possède un niveau équivalent à celui d’une ou des chaînes sous-jacentes, il est donc pas nécessaire de faire confiance à une entité externe aux \textit{blockchains}.  La notion  d’extensibilité signifie que le \textit{bridge} est compatible un grand nombre de chaînes.
Un \textit{bridge} respecte la notion de généralisation s’il est capable d'échanger n’importe quel type de données accepté par les deux chaînes.\\

Pour illustrer ces termes, il est possible de les appliquer aux types de vérification appartenant aux \textit{bridges}. La vérification locale respecte les notions d’extensibilité et de \textit{trustless} puisque qu'elle est applicable sur tous les \textit{bridges} peu importe les chaînes reliées et le niveau de fiabilité dépend de la chaîne la plus faible.
La vérification native ne respecte pas la notion d’extensibilité car le \textit{bridge} n’est pas réutilisable. Néanmoins elle respecte la notion de généralisation parce qu’elle est codée de manière spécifique aux \textit{\gls{blockchain}s} reliées au \textit{bridge}. Le critère basé sur la notion \textit{trustless} est également rempli étant donné que le niveau de fiabilité dépend des \gls{vérificateur}s des chaînes.
La vérification externe ne respecte pas la notion de \textit{trustless} cependant elle est fortement extensible et générale vis-à-vis des données. \cite{Ngrave}

 Suite au paragraphe précédent, il est possible de constater que les \textit{bridges} interopérables ne respectent que deux des trois notions énoncées. Ce problème est connu sous le nom de trilemme de l’interopérabilité. 

\subsubsection{Une solution optimiste}

Une solution proposée pour résoudre ce trilemme est un \textit{optimistic bridge}(\textit{bridge} optimiste) nommé vis-à-vis de sa vérification portant le même nom\cite{OptimisticBhuptani}. En effet, contrairement aux \textit{bridges} vérifiés de manière native, locale ou externe, la vérification optimiste dépend de l’utilisation d’une latence lors de la confirmation du transfert des \gls{actif}s entre les \textit{\gls{blockchain}s}. Cela priorise donc la sécurité au détriment de la vivacité, puisque les transactions sont par conséquent plus lentes mais sécurisées car le \textit{bridge} est \textit{trustless}. \\

Voici plus en détails le déroulement du processus de vérification optimiste d’un \textit{bridge}.
Ceci commence par l’envoi d'une demande de transaction de la part d’un utilisateur ou d’une application décentralisée vers la \textit{\gls{blockchain}} native par le biais d’une fonction contrat. Cette demande est ensuite acceptée par un validateur (une entité équivalente à un \gls{vérificateur}, la seule différence étant leur rôle) puis inscrite sur un \textit{block} de la chaîne.\cite{NomadDocsVerification}
Ensuite un \gls{vérificateur} a pour rôle de vérifier la transaction. Pour cela, le \gls{vérificateur} signe le haché des données envoyées précédemment. 
Suite à cela, n’importe quel système de relais peut lire le haché signé sur la chaîne originelle et l’inscrire sur une ou plusieurs chaînes destinataire. Les validateurs de la chaîne destinataire valide et l'inscrivent sur leur chaîne. Cette action déclenche alors une latence de trente minutes pendant laquelle un observateur peut signaler et prouver une fraude effectuée par la chaîne native ce qui déconnectera la communication avec la chaîne destinataire. 

Deux scénarios sont alors possibles. Premier cas, si aucun observateur ne se manifeste, les données de la transaction sont finalisées puis traitées par la chaîne destinataire. Les validateurs sont récompensés pour leur travail avec une partie des frais de transaction. \cite{Fees} Deuxième cas, un observateur prouve une fraude pendant les trente minutes accordées. Le \gls{vérificateur} ayant fraudé est pénalisé par la perte de sa récompense (qui sera obtenu par l'observateur) et son exclusion du réseau.\cite{EthereumSlashing}

\subsubsection{Possibles faiblesses de l’optimisme et leurs solutions}

Le bon fonctionnement des \textit{bridges} optimistes dépend des chaînes auquelles il est rattaché donc tant que ces dernières sont protégées correctement la seule conséquence d’une faille du \textit{bridge} est l’arrêt système plutôt qu’une perte de fonds comme cela peut être le cas avec les autres types de \textit{bridge}.

Les deux acteurs principaux ayant les moyens de nuire au bon fonctionnement du \textit{bridge} ainsi que de la transaction est l’agent \gls{vérificateur} ainsi que l’observateur car ces deux rôles ont de l’influence sur cette dernière. \\

Le premier cas impliquant le \gls{vérificateur} se nommant \textit{Updater Fraud} (fraude du \gls{vérificateur})  fut déjà mentionné lors de la présentation du fonctionnement du \textit{bridge} optimiste. Ce dernier repose sur le fait que toute transaction doit passer par le \gls{vérificateur} et que par conséquent toute fraude est originaire de ce dernier. Sinon si cela venait d’un autre participant, le \gls{vérificateur} n’aurait alors pas accepté la transaction. C’est pourquoi, lors de l’intervention d’un observateur prouvant une fraude, le \gls{vérificateur} est sanctionné par le retrait sur son solde d'un montant équivalent à la récompense promise et par son exclusion du réseau de la \textit{blockchain}. \\

La seconde faiblesse liée aux \gls{vérificateur}s est un \textit{Updater DoS} ou déni de services de la part du \gls{vérificateur}. En effet, il est possible que le processus soit interrompu si un validateur arrête de signer empêchant l'échange inter-chaînes de se produire.
Une solution a été implémentée pour palier à cela comme la mise en place d’un système de substitution avec la présence de plusieurs \gls{vérificateur}s sur une même chaîne afin de pouvoir prendre le relai en cas de manque de réponse de la part de celui étant rattaché au transfert.  Pour éviter que ce scénario se produise fréquemment le \gls{vérificateur} ayant manqué son tour lors de la signature (que cela soit accidentel ou voulu) est pénalisé de la même manière que le cas précédent. \\

Maintenant que les possibles obstacles au bon fonctionnement du \textit{bridge} liés aux \gls{vérificateur}s ont été mis en lumière, il est également possible que l’observateur ait un comportement malveillant.  Effectivement, malgré l’absence de tromperie (puisque le \gls{vérificateur} remplit son rôle), l’observateur peut abuser du mécanisme de déclaration de fraude pour impacter le bon déroulement du procédé. \\

La faculté de l’observateur à pouvoir couper la connexion s’il conteste la transaction lui permet d’effectuer un déni de service appelé \textit{Watcher DoS}. C’est pourquoi il lui ait possible de fermer définitivement la connexion d’une transaction si ce dernier continue sans cesse de couper le processus sans raison valable. Heureusement, la fermeture ne concerne que la connexion et n’impacte en aucun cas le système du \textit{bridge}. Cependant cette attaque semble irrationnelle en terme de ressources et de temps car l’observateur effectuant le déni de service ne gagne rien financièrement contrairement au processus habituel. En effet, si un observateur prouve une fraude correctement, ce dernier peut récupérer la récompense du \gls{vérificateur}. Mais ici puisqu’aucune fraude n’est prouvée les données se trouvant sur la chaîne d’origine sont conservées et sécurisés. Cela cause seulement une perte de temps pour l’utilisateur ou l’application décentralisée voulant effectuer l'échange d’une \textit{\gls{blockchain}} à une autre.

Une réponse à ce problème actuellement mise en œuvre par le \textit{bridge} de \gls{Nomad} est la présence d’un groupe restreint d’observateurs autorisés à contester, de cette manière il est facile de connaître les observateurs malveillants. Chaque observateur possède une clé permettant de signer une attestation confirmant la présence d’une fraude dans la transaction, chaque \textit{bridge} stocke un ensemble contenant les adresses des attestations appartenant aux observateurs autorisés. Si l’attestation reçue par le \textit{bridge} est présente dans l’ensemble alors la connexion est rompue\cite{NomadDocsWatcher}.
Sur le long terme, une proposition consistant à la mise en place de frais si l’on souhaite contester est en train d’être étudiée. Le montant doit répondre à deux contraintes: ce dernier doit être assez haut pour dissuader les observateurs malhonnêtes mais assez bas pour que ceux ayant réellement l’envie de prouver de manière valide une fraude existante puissent le faire. Dans la continuité de cette solution, il serait également possible de récupérer la signature de la déconnexion générée par l’observateur sur la chaîne originale et de le pénaliser en lui retirant les frais qu’il a payé tel une garantie\cite{OptimisticBhuptani}.



\subsection{Wormhole}
En 2017, une cryptomonnaie adossée à la \textit{blockchain} Solana a émergée avec des caractéristiques
similaires à Ethereum : \textit{blockchain} publique, \textit{smart contracts}.\\
Solana est devenue de facto une \textit{blockchain} concurrente à Ethereum et est aujourd'hui 
la onzième \textit{blockchain} en terme de capitalisation selon l'aggrégateur de marché Coinmarketcap.\\
Un besoin d'échanger des actifs entre les \textit{blockchains} Ethereum et Solana est apparu, 
d'où l'introduction en 2020 de la première version de Wormhole.
Initialement, Wormhole v1 a été concu comme un \textit{bridge} entre Ethereum et Solana.
Depuis, Wormhole s'est développé au-delà de Solana avec le lancement d'une deuxième version en 2021 
en tant que protocole générique de passage de messages.\\
À l'écriture de ce rapport, 22 \cite{wormholeNetwork} \textit{blockchains} sont compatibles avec Wormhole 
dont : BNBChain, Ethereum, Moonbeam, Polygon, Solana...\\
Le protocole émet un message à partir d'une \textit{blockchain} source qui est validé par un réseau de 
gardiens.\\ 
Le message est ensuite envoyé à la \textit{blockchain} cible pour être traité.

\subsubsection{VAA (\textit{Verified action approval})}

Lorsqu'un \textit{smart contract} envoie un message \textit{crosschain} comme un verrouillage
de jetons sur une \textit{blockchain} source et une demande de frappe de jetons sur une 
\textit{blockchain} cible, celui-ci interargit avec un \textit{core contract} \cite{wormholeCoreContract}.
Un \textit{core contract} est déployé sur toutes les \textit{blockchains} compatibles avec le protocole 
Wormhole. Tout \textit{core contract} est observé par le réseau de gardiens.
Un message Wormhole est émis grâce à la fonction \textit{publishMessage()} prenant en entrée le \textit{payload}.
La sortie de cette fonction est un \textit{sequence number}, un numéro d'index unique pour le message.
Combiné à l'adresse du contrat de l'émetteur et à l'identifiant de la chaîne de l'émetteur, le message 
correspondant peut être récupéré auprès d'un nœud du réseau de gardiens.\\
Un message Wormhole est vérifié grâce à la fonction \textit{parseAndVerifyVAA()} prenant en entrée le message.
Selon la validité de l'entrée, la fonction retourne en sortie le \textit{payload} ou une exception.
\newpage

VAA \cite{wormholeVAA} est la primitive de messagerie de base de Wormhole. Un VAA contient une en-tête 
ainsi qu'un \textit{body}. L'en-tête contient l'index des gardiens ayant signés le message et la collection des signatures.
L'en-tête permet au \textit{core contract} de vérifier l'authenticité du VAA.
Quant au \textit{body}, il contient des informations comme le numéro d'identification de la chaîne 
Wormhole du contrat émetteur, l'adresse du contrat émetteur, le \textit{sequence number} 
et le \textit{payload}.\\ 5 \textit{payloads} peuvent être utilisés dont \textit{Transfer} et 
\textit{AssetMeta}, attestant les méta-données du jeton.\\
Le \textit{payload AssetMeta} est obligatoire avant un premier transfert.
En effet, le \textit{payload Transfer} n'informe pas la chaîne B des meta-données du jeton verrouillé.
En l'absence de connaissance de ces informations, il n'est pas possible pour la \textit{blockchain} B 
de frapper la quantité correcte de jetons.\\
Si l'on souhaite ensuite transférer des jetons depuis une \textit{blockchain} A vers une 
\textit{blockchain} B, il faut verrouiller les jetons sur A et les frapper sur B.
D'où l'utilisation du \textit{payload Transfer} contenant des informations comme la 
quantité de jetons transférés, l'adresse de la chaîne d'origine et de destination, 
le numéro d'identification de la chaîne d'origine et de destination..
Une preuve doit être fournie que les jetons sur A sont verrouillés avant que la frappe puisse 
avoir lieu sur B. La signature des gardiens sur le VAA correspondant est la preuve apportée à 
la \textit{blockchain} B que le verrouillage a bien été effectué et que la frappe de jetons sur 
B est légitime.

\subsubsection{Gardiens}

Un gardien \cite{wormholeGuardian} est une autorité de confiance qui a comme de rôle valider 
(par une signature) le \textit{payload} contenu dans un VAA.
Comme évoqué précédemment, le réseau de gardiens observe tous les messages \textit{crosschain} via la 
surveillance des \textit{core contracts}.
Le réseau de gardiens est composé de 19 gardiens à parts égales sans chef (\textit{leaderless}).
Il est conçu pour servir d'oracle à Wormhole et est l'élement le plus critique de l'écosystème.
Si une majorité de deux tiers ou plus des gardiens signent le même VAA, alors le consensus est atteint : 
le VAA est automatiquement considéré valide par  tous les contrats Wormhole sur toutes les 
\textit{blockchains} et le \textit{payload} est actionné. 
Chaque gardien utilise un algorithme de signature à courbe elliptique : ECSDA pour 
\textit{Elliptic Curve Signature Digital Algorithm}.
Plus précisément, chaque gardien se réfère à «secp256k1» comme paramètres de la courbe elliptique, 
aussi utilisé par les \textit{blockchains} Bitcoin et Ethereum.\\
Le modèle de consensus utilisé est une \textit{Proof of Authority} (PoA) avec un système de 
\textit{multisignature} M/N \cite{wormholeChainswap}, c'est à dire que M clefs parmi N sont nécessaires 
pour signer un VAA. Ce modèle permet un traitement rapide des transactions et une dispense de participation monétaire, par rapport à la preuve de travail (PoW) et la preuve 
de participation (PoS). Cependant, il présente également des désavantages : le système est par 
\textit{design} centralisé et dépend d'un petit groupe de nœuds pouvant créer un point de 
défaillance unique par l'utilisation commune d'une fonction vulnérable. Il est questionnable de restaurer des tiers de confiance dans le cadre d'un système 
devenu populaire grâce à l'absence de tels autorités. Wormhole justifie la décentralisation de leur 
système \cite{wormholeGuardian} par la présence de plusieurs parties (et non d'un seul) dans le contrôle du réseau.
Selon notre analyse, la décentralisation résulte de l'absence d'un ou plusieurs tier(s) de confiance lorsque deux parties 
souhaitent réaliser une transaction.
\newpage

\subsubsection{Relais}

Un relai \cite{wormholeRelayer} est un processus qui délivre un ou plusieurs VAA(s) à une destination.
Les relais ne sont ni de confiance, ni privilégiés, ils écoutent directement le réseau de gardiens 
via l'intermédiaire d'un processus espion. Ces relais ne peuvent pas compromettre l'intégrité d'un VAA 
car une altération serait détectée lors du processus de vérification des signatures. Cependant, il n'est 
pas assuré qu'un relai transmette un VAA à destination, d'où une perte de disponibilité. Il est conseillé
d'héberger soi-même ces relais pour supporter son application.

\begin{figure}[h!]
    \centering
    \includegraphics[scale=0.5]{centralisation/uml_design_v2.png}
    \label{fig:wormholeDesign}
    \caption{Architecture Wormhole \cite{wormholeArch}}
  \end{figure}

% @startuml
% rectangle r1 as "Source Token Bridge 
% Relayer Contract"
% rectangle r2 as "Source Token 
% Bridge"
% rectangle r3 as "Source Core 
% Contract"
% storage "Guardians" as r4
% hexagon r5 as "Off-chain 
% Message Relayer" 
% rectangle r6 as "Target Token Bridge 
% Relayer Contract"
% rectangle r7 as "Target Token 
% Bridge"
% rectangle r8 as "Target  Core 
% Contract"

% r1 -> r2 : Transfer()
% r2 -> r3 : publishMessage()
% r3 -do-> r4 : Guardien reads message
% r4 -left-> r5 : Signed message

% r5 -do-> r6 : Signed message
% r6 -> r5 : Relayer fee

% r6 -> r7 : Signed message
% r7 -> r6 : Wrapped token

% r7 -> r8 : parseAndVerifyVAA()
% @enduml




\subsection{Analyse d'attaques}
% Auteur Romain TESTUD
\subsubsection{Mise en contexte}
Les \textit{\gls{blockchain}s} et leurs protocoles d'échanges ne sont pas exemptes d'attaques informatiques ou bien de défaillances.
Ces attaques peuvent cibler des portefeuilles (attaques sur des \textit{hot wallets}\footnote{portefeuille de cryptomonnaies en ligne, à différencier des \textit{Cold Wallets}, des portefeuilles hors lignes}) ou encore des \textit{bridges}. 
Ce sont ces dernières qui nous ont intéressées dans le cadre de ce projet de recherche sur les échanges inter-\gls{blockchain}s. 
Les bridges, comme explicité dans la partie dédiée du rapport, sont des protocoles permettant la circulation de données entre deux \textit{\gls{blockchain}s} différentes.\\
Nous avons, au cours de nos recherches, trouvés de nombreux cas d'attaques sur des protocoles d'échanges inter-\gls{blockchain}s. 
De manière à illustrer les types d'attaques possibles et les points critiques de ces systèmes nous allons décrire deux attaques parmi les plus importantes : \textit{Wormhole} et \textit{Nomad}.

\subsubsection{Le cas Wormhole}
Nous vous avons présenté le protocole \textit{Wormhole} dans la partie précédente. 
Le 2 Février 2022, une attaque exploite une erreur d'implémentation dans une \textit{\gls{dApp}} sur la chaîne Solana \cite{SolMed} \cite{SolRekt}. 
Pour se faire l'attaquant à réussi à contourner la vérification des signatures des gardiens en exploitant
une correction de bug ayant été publié sur le code source du projet mais n'étant pas encore effective en production.
Ainsi il à réussi à récupérer l'équivalent de 120 000 \textit{ETH} en \textit{whETH} (\textit{Wormhole ETH}). 
Lors d'un transfert de jetons d'une chaîne à une autre, plusieurs étapes sont réalisées par différentes fonctions.
Après la formulation de la transaction, une fonction va se charger de récupérer les signatures des gardiens dans un \textit{SignatureSet}\footnote{Ensemble de signatures de gardiens}, ces dernières sont ensuite vérifiées. 
Pour cela, une fonction nommée \texttt{verify\_signature} va appeler un programme de vérification de Solana permettant l'analyse du \textit{SignatureSet}. 
L'appel à ce programme se fait de la manière suivante, en utilisant le nom \texttt{sysvarinstruction} \cite{SolGitError} dans la transaction. 
Dès lors que les signatures sont validées, un \textit{VAA} peut être émis et transmis vers la \textit{\gls{blockchain}} souhaitée. \\
La transaction de l'attaquant étant frauduleuse, il n'aurait donc pas pu obtenir de signatures des gardiens. 
Pour contourner cette étape de récupération des signatures la transaction de l'attaquant était dotée d'un \textit{SignatureSet} correspondant à une transaction antérieure. 
Seulement, n'étant pas pour la bonne opération cet ensemble ne peut pas être approuvé par \texttt{verify\_signature}. 
C'est ici que l'attaquant à utilisé un défaut d'implémentation pour valider son \textit{SignatureSet}. 
Comme décrit précédemment, la fonction \texttt{verify\_signature} appelle un programme pour effectuer la vérification des signatures. 
Cependant il n'y à pas de vérification faites sur le programme appelé, l'attaquant à pu donc utiliser une adresse différente lui permettant de valider sa transaction. 
Avec le  \texttt{SignatureSet} ainsi validé, l'attaquant a pu générer un \textit{VAA} valide et pu déclencher une frappe de jeton vers son propre compte sans en avoir déposé au préalable. 
La correction de cette faille était contenue dans la mise à jour évoquée en début de paragraphe\cite{SolGitFixed}, permettant la vérification du programme appelé pour la vérification. 

\subsubsection{Le cas Nomad}
Nomad est un protocole d'interopérabilité entre chaînes permettant de passer des \gls{actif}s entre deux \textit{\gls{blockchain}s} différentes. 
Pour fonctionner, ce protocole fait appel à des applications décentralisées opérant sur les chaînes du réseau. 
Une première \textit{\gls{dApp}} appelée \textit{réplica} est déployée sur les \textit{\gls{blockchain}s} recevant les messages, elle fait office de "boite de réception". 
Une seconde \textit{\gls{dApp}} appelée \textit{home} est déployé sur les \textit{\gls{blockchain}s} émettrices de message. \\
Le 1\textsuperscript{er} août 2022 une attaque exploitant une erreur d'implémentation sur l'application \textit{Réplica} a engendré une perte de 190 millions de dollars en liquidité \cite{NomadMedium} \cite{NomadRekt}.
Cette attaque s'est déroulée après le déploiement d'une mise à jour, un moyen de contourner la vérification des signatures du message étant apparu. 
En analysant l'application \textit{Réplica} après la mise à jour, nous pouvons voir que lors d'une initialisation, la racine des messages, appelée \texttt{\_commitedRoot}, est initialisée à $0$, ce signifiant que le message n'a pas encore été validé. 
\begin{lstlisting}[caption={Fonction \textit{initialize} de \textit{Réplica} contenant une erreur \cite{NomadGitError}}]
    function initialize(
        uint32 _remoteDomain,
        address _updater,
        bytes32 _committedRoot,
        uint256 _optimisticSeconds
    ) public initializer {
        __NomadBase_initialize(_updater);
        // set storage variables
        entered = 1;
        remoteDomain = _remoteDomain;
        committedRoot = _committedRoot;
        // pre-approve the committed root.
        confirmAt[_committedRoot] = 1;
        _setOptimisticTimeout(_optimisticSeconds);
    }
\end{lstlisting}

Dans les lignes précédentes nous observons cette affectation : \texttt{confirmAt[\_commitedRoot] = 1}, le rôle de cette ligne est de pré-approuver la racine d'un message. 
Cette fonction est utilisée pour approuver le premier message lors du déploiement du contrat sur une \textit{\gls{blockchain}}. 
Or ici, la valeur de la racine à été initialisée a $0$, donc cette racine devient une racine valide pour la fonction de vérification des messages. 
Seulement comme nous l'avons énoncé précédemment, $0$ est la valeur par défaut pour un message n'ayant pas encore été vérifié. 
Ainsi, lors de l'émission d'un message par la fonction \texttt{process}, tout message non vérifié sera envoyé. 
Cette erreur d'implémentation a permis à des pirates d'effectuer plusieurs transactions frauduleuses et de retirer l'équivalent de 190 Millions de dollars dans la réserve de liquidité du bridge de Nomad. 
Le contrat à été corrigé, dans une mise en ligne datant du 3 Septembre 2022, tel que la racine $0$ n'est plus pré-approuvée. 

\begin{lstlisting}[caption={Fonction corrigée de l'application \textit{Réplica} \cite{NomadGitFixed}}]
    function initialize(
        uint32 _remoteDomain,
        address _updater,
        bytes32 _committedRoot,
        uint256 _optimisticSeconds
    ) public initializer {
        __NomadBase_initialize(_updater);
        // set storage variables
        entered = 1;
        remoteDomain = _remoteDomain;
        committedRoot = _committedRoot;
        // pre-approve the committed root.
        if (_committedRoot != bytes32(0)) confirmAt[_committedRoot] = 1;
        _setOptimisticTimeout(_optimisticSeconds);
    }
\end{lstlisting}


%\subsubsection{DeFi hacklabs}
%Lors de nos recherches sur des attaques sur des protocoles inter-\gls{blockchain}s, nous avons découvert \textit{Web3sec}, un groupe de recherche centré sur la sécurité du web3. 
%Le groupe met à disposition des ressources indexés sur une page notion (en annexe) :
%\begin{itemize}
%    \item Plusieurs dépots \textit{Github} pour étudier les attaques et apprendre les vulnérabilités sur ces types de programmes.
%    \item \textit{DeFi Hacks Analysis - Root Cause} : Une base de données d'analyses d'attaques sur des solutions et organismes traitant sur des \gls{blockchain}s, les analyses sont sourcées et redirigent vers le dépot \textit{GitHub} de reproduction des attaques.
%    \item Un ensemble d'outils utilitaires tels que des outils de debug de transaction, des dashboards ou encore des newletters.
%\end{itemize}
%Ces outils nous ont permis d'explorer de nombreuses sources concernant les attaques sur les protocoles inter\gls{blockchain}s.

\subsubsection{Contre-mesures et solutions envisageables}
Comme nous avons pu le voir, de nombreux cas d'attaques sont observables sur des protocoles d'échanges centralisés. 
Dans la plupart des cas, elles résultent de problèmes d'implémentations et autres oublis dans les codes sources des protocoles utilisés. 
Cela peut être expliqué par le fait que la \textit{\gls{blockchain}} est un domaine qui évolue très vite et chaque innovation technique peut rapporter des parts de marché importantes au premier arrivé. 
De plus, de nombreux acteurs se spécialisent dans ce domaine sans nécessairement avoir une grande culture de la cybersécurité.
Il se peut donc que des erreurs d'implémentations paraissant évidentes ne soient pas relevés lors de la mise en production. \\ 
Des moyens de limiter le plus possible l'apparition de tels événements sont néanmoins possibles. 
Tout d'abord des standards de sécurité pourraient être mis en place afin de déterminer un socle minimal à atteindre. 
Des audits et des analyses de sécurité peuvent être mis en place pendant la production ou après la publication des protocoles, en respectant un cycle de développement "classique".
%Nous pouvons aussi nous questionner quant à l'implication du tier de confiance dans l'apparition de ces failles. 
%En effet, les attaques les plus importantes ont comme point d'entrée une faille dans la structure de l'intermédiaire. 
%Cette menace constante d'attaques a mené à des recherches visant à soutirer ce tiers des échanges inter-\gls{blockchain}. Ce menant aux échanges décentralisés.


\subsection{Les limites du centralisé}
% Auteur : Romain TESTUD
\begin{frame}{Les limites du centralisé}
    \begin{itemize}
        \item Manque de transparence.
        \item Manque de maturité en matière de sécurité.
        \item Questionnement sur le tiers de confiance
    \end{itemize}
    \pause
    \begin{block}{Le tiers de confiance}
        \begin{itemize}
            \item Source d'attaques.
            \item Point critique des CEX.
        \end{itemize}
        $\rightarrow$ Volonté de se passer d'un tiers de confiance.
    \end{block}
\end{frame}

\end{document}