% propos introductif
Les échanges financiers sur Internet reposent presque exclusivement via les institutions financières, qui
agissent comme tiers de confiance pour le traitement des paiements électroniques.
En 2008, le \textit{whitepaper Bitcoin} a partagé une solution permettant à deux parties d'échanger de
la monnaie électronique. La particularité de cette solution est la suppression de ce modèle de confiance par
l'ajout d'une preuve cryptographique. Cela permet de s'émanciper de la centralisation exercée par ces 
institutions et de tendre vers la décentralisation.

% \textit{\gls{blockchain}}, fonction de hachage cryptographique, immuable
La structure de donnée sous-jacente est la \textit{\gls{blockchain}}, une base de données distribuée constituée 
d’une chaîne de blocs liés et sécurisés par des hachés cryptographiques. Une des propriétés des fonctions de hachage 
est qu'une modification de l'entrée modifie la sortie, le haché.
D'où le fait qu'une \textit{\gls{blockchain}} est considérée comme immuable , chaque bloc (sauf le premier) 
est lié au bloc précédent car il contient le haché de ce dernier. Toute tentative de modifier un bloc antérieur 
affecterait tous les blocs suivants, créant ainsi une incohérence dans la chaîne.

%distribué, réseau de noeuds
Dans un système centralisé, toutes les transactions sont enregistrées dans une base de données unique gérée 
par une entité centrale telle qu'une banque. Cette entité est responsable de l'intégrité des données
et les usagers font confiance que toute modification involontaire ou malveillante sera détectée. 
Dans notre cas, une \textit{\gls{blockchain}} est distribuée, un réseau de noeuds connectés travaillent ensemble pour valider et enregistrer les blocs. 
Un noeud est un périphérique connecté à un réseau pair à pair qui stocke une copie de la \textit{\gls{blockchain}} et participe à la validation 
et à la propagation des blocs. En raison de sa nature distribuée, les noeuds peuvent valider et ajouter un bloc sans avoir 
recours à un tiers de confiance, ce qui accrût son caractère décentralisé.

% l'économie de la \textit{\gls{blockchain}}
 Depuis, la \textit{\gls{blockchain}} a grandi en popularité et de nombreuses \textit{\gls{blockchain}}s sont apparus.
On peut dénoter la \textit{\gls{blockchain}} Ethereum qui a introduit la notion de \textit{\gls{smart contract}} et de jetons.
Un besoin d'échange d'actifs entre \textit{\gls{blockchain}}s est apparu, d'où l'implémentation de protocoles 
d'échanges de jetons. L'objectif de ce rapport est de dresser un état de l'art de ces protocoles 
en deux temps : les protocoles d'échanges sur les plateformes centralisées puis les protoles d'échanges 
décentralisés.

% Le besoin d'implémenter des protocoles d'échanges de jetons inter-blockchain a ajouté un maillon supplémentaire à sécuriser
% vulnérable et pouvant compromettre la sûreté et la sécurité des blockchains à ces bridges / protocoles

% Une problématique est l’échange de jetons entre deux ou plusieurs block-
% chain. Des attaques spectaculaires sur des blockchains ces dernières années ex-
% ploitent des faiblesses dans l’implémentation ou la conception des protocoles
% d’échange [3, 4, 5]
% L’objectif de ce ter est de dresser un panorama des protocoles d’échange de
% jetons entre entre blockchains ainsi que de leurs différentes faiblesses

% Il s’agit de produire un état l’art sur les protocoles d’échanges de jetons entre
% blockchains.
% — On pourra dans un premier temps s’intéresser aux systèmes centralisés,
% qui nécessitent une plateforme tiers pour réaliser l’échange. Pour en ap-
% préhender les principes, on pourra commencer par étudier les analyses
% d’attaques sur les plateformes [3, 4, 5].
% — Ensuite, avec comme point de départ les articles [1, 2] on s’intéressera aux
% protocoles d’échange centralisés. En fonction du temps restant, le groupe
% pourra implémenter un ou plusiers protocoles proposés dans [1, 2] ou
% d’autres découverts au cours de l’étude.
