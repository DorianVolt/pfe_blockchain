\subsection{Centralisé} 
Les échanges centralisés sont les plus utilisés dans les environnements \textit{\gls{blockchain}}. Ils offrent en effet
des avantages tels qu'une grande accessibilité ainsi qu'un large panel de produits financiers connexes.
De plus, le marché de l'échange centralisé dispose d'une grande diversité d'acteurs et de plateformes, ce
qui laisse un choix à l'utilisateur et a pour effet de stimuler le marché. 
Cependant, ils présentent aussi des inconvénients. Premièrement, les acteurs de l'échange centralisé entretiennent une certaine opacité 
de leurs algorithmes et  de leurs protocoles, ce qui est un frein à l'utilisateur final pour comprendre et analyser 
les transactions réalisées. Ceci l'oblige donc à faire confiance en la plateforme qu'il utilise. 
De plus, l’utilisation d’un tiers de confiance peut entraîner des coûts supplémentaires pour 
les utilisateurs et a pour effet de rajouter un point critique en terme de sécurité. Enfin, ces plateformes ont 
des obligations légales concernant la collecte de données personnelles ce qui entraîne une violation de la vie 
privée des utilisateurs. \\
En fin de compte, les solutions centralisées ne s’inscrivent pas dans l’idéologie initiale de la \textit{blockchain}, qui est 
basée sur la décentralisation et la transparence.

\subsection{Décentralisé}
Les solutions décentralisées offrent une alternative aux échanges centralisés en permettant aux utilisateurs 
d’échanger directement entre eux sans avoir besoin d’un tiers de confiance. Cela signifie que les utilisateurs 
ont un contrôle total sur leurs transactions et qu’ils n’ont pas à faire confiance à une plateforme tierce pour 
effectuer des transactions. Les solutions décentralisées offrent également d'autres avantages. En terme de 
transparence dans un premier temps. En effet, il est simple pour l'utilisateur de visualiser exactement comment les 
transactions sont effectuées et il est souvent simple de construire un écosystème communautaire autour d'un projet
décentralisé. \\
Les systèmes décentralisés offrent aussi une plus grande fiabilité de leurs services, puisqu'étant distribués 
le systèmes est plus résilient aux pannes ce qui offre une plus grande garantie dans les transactions.

Cependant, les solutions décentralisées ont également des inconvénients. Tout d’abord, elles peuvent être 
difficiles à prendre en main pour les utilisateurs qui ne sont pas familiers avec la technologie \textit{blockchain}. 
De plus, elles peuvent avoir un intérêt économique moins important pour les investisseurs car il est plus
difficile de créer un modèle économique autour de ces solutions.

\subsection{Générale}
Pour conclure le sujet, nous constatons un certain flou entre ce qui est considéré comme centralisé et décentralisé. 
Durant nos recherches, nous avons pris comme référentiel une certaine définition de ce que nous considérons comme
centralisé ou décentralisé. Il nous parait important de souligner que l'usage de ces termes présente un argument \textit{marketting}
important, et beaucoup d'acteurs remanient la définition de décentralisé pour correspondre avec leurs produits.

% \begin{table}[h!]
%     \centering
%     \caption{Tableau Récapitulatif}
%     \begin{tabular}{|l|l|l|}
%     \hline
%                   & Centralisé                                                                                                                                                      & Décentralisé                                                                                                 \\ \hline
%     Avantages     & \begin{tabular}[c]{@{}l@{}}Facilité d'utilisation\\ Popularité\\ Variétés des produits financiers\end{tabular}                                                  & \begin{tabular}[c]{@{}l@{}}Fiabilité et Sécurité\\ Maitrise des données\\ Frais de transactions\end{tabular} \\ \hline
%     Inconvénients & \begin{tabular}[c]{@{}l@{}}Sécurité et Fiabilité relative à la plateforme\\ Frais de transactions\\ Opacité des plateformes\\ Dépendance aux tiers\end{tabular} & \begin{tabular}[c]{@{}l@{}}Complexité d'implémentation\\ Difficulté d'usage\end{tabular}                     \\ \hline
%     \end{tabular}
% \end{table}


\begin{table}[h!]
    \centering
    \caption{Tableau Récapitulatif}
    \begin{tabular}{|l|l|c|c|}
        \hline
                                   &                        & \multicolumn{1}{l}{Décentralisation}                   & \multicolumn{1}{l|}{Accessibilité}                  \\  \hline
                                   & Plateformes d'échanges & \cellcolor[HTML]{FD6864}$---$                          & \cellcolor[HTML]{9AFF99}$+++$      \\ \cline{2-4}
    \multirow{-2}{*}{Centralisé}   & Blockchains Bridges    & \cellcolor[HTML]{FD6864}$-$                            & \cellcolor[HTML]{9AFF99}$++$      \\ \hline
                                   & Relay                  & \cellcolor[HTML]{9AFF99}$+$                            & \cellcolor[HTML]{FD6864}$--$                              \\ \cline{2-4}
                                   & Sidechains             & \cellcolor[HTML]{9AFF99}$++$                           & \cellcolor[HTML]{FD6864}$--$                              \\ \cline{2-4}
                                   & Reserves de Liquidités & \cellcolor[HTML]{9AFF99}$+++$                          & \cellcolor[HTML]{9AFF99}$+++$      \\  \cline{2-4}
                                   & HTLC                   & \cellcolor[HTML]{9AFF99}$+++$                          & \cellcolor[HTML]{FD6864}$---$                             \\ \cline{2-4}
    \multirow{-5}{*}{Décentralisé} & Off-chain              & \cellcolor[HTML]{9AFF99}$++$                           & \cellcolor[HTML]{9AFF99}$+$       \\ \hline
    \end{tabular}
    \end{table}